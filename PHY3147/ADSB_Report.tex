% Comments start with % (percent) character and last till the end of the line.
%
% The line below tells TeXworks editor to use pdflatex for compilation
% of this document; remove it if you want to use another engine 
%
%!TEX program = pdflatex
%
% LaTeX2e document starts with \documentclass[options]{<class-name>}
% <class-name> can be one of the standard LaTeX document classes: 
% article, report or book, or some other specialised class.
%
\documentclass{article}
%
% Preamble of LaTeX document is everything before \begin{document}.
% Preamble is used to load extension packages and to set up global 
% parameters and configuration for the entire document.
%
% Extension packages providing additional functionality
\usepackage{amsmath}       % additional math environments
\usepackage{graphicx}      % graphics import from external files 
\usepackage{epstopdf}      % automates .eps to .pdf conversion
% epstopdf package may require --shell-escape option to pdflatex
\usepackage{booktabs}      % table typesetting additions
\usepackage{siunitx}       % number and units formatting
\usepackage{caption}       % customisation of captions
\usepackage{subcaption} 
\usepackage{url}           % format url addresses
\usepackage{abstract}		% allows formatting of abstract
\usepackage[margin=0.5in]{geometry}

%\usepackage{tikz,pgfplots} % diagrams and data plots
%
% set up caption options
\captionsetup{margin=12pt,font=small,labelfont=bf}
%
%removes abstract title

\renewcommand{\abstractname}{Abstract}
\renewcommand{\abstractnamefont}{\normalfont\Large\bfseries}

%sets abstract margins
\setlength{\absleftindent}{10mm}
\setlength{\absrightindent}{10mm}

%
% global options for siunitx
%\sisetup{seperr,repeatunits=false,per=symbol}
%
% some handy commands for referencing;
% the optional argument overrides the default label, e.g.
% \figref[FIG.~]{fig:label}
\newcommand{\figref}[2][\figurename~]{#1\ref{#2}}
\newcommand{\tabref}[2][\tablename~]{#1\ref{#2}}
\newcommand{\secref}[2][Section~]{#1\ref{#2}}


% The document content starts with \begin{document} 
% and is finished with \end{document}
%

\title{\textbf{Predicting the Weather by Watching Aeroplanes: 
\newline
Applying Refractive Techniques to Air-plane ADSB Radio Wave Signals in the Lower Troposphere for Determining Relative Humidities.}} % fill in the title here
\author{Matthew Evans}% fill in your name here
\date{27\textsuperscript{th} March 2019} % date of the report
\usepackage{titling} % centralises the title
\renewcommand\maketitlehooka{\null\mbox{}\vfill}
\renewcommand\maketitlehookd{\vfill\null}
\thispagestyle{empty}

\begin{document}

%\twocolumn[	% makes title and abstract appear over entire page width
\maketitle % formats the title
\thispagestyle{empty}
\newpage

\vspace*{\fill}
\begin{abstract}

\noindent
Various wave properties were investigated by exploring water waves in a ripple tank and studying the diffraction of laser light. A value of $ 0.18 \pm 0.05$ ms$^{-1}$ for water wave speed at varying frequencies and a fixed depth of $h = 1.0 \pm 0.3$ cm was obtained and the one generated by using $v \approx \sqrt{gh}$ \cite{Book02} was $ 0.3 \pm 0.1$ ms$^{-1}$. The angles of incidence and reflection for the reflection of water waves at a barrier were found to be $41 \pm 2^{\circ}$ and $41 \pm 2^{\circ}$ respectively. The water wave speed was also investigated at a fixed frequency of $20$ Hz with varying depths allowing a value of $4.4 \pm 0.1$ ms$^{-2}$ for the acceleration due to gravity to be determined with the known value as 9.8 ms$^{-2}$. The value for the wave speed at varying frequencies was different to the one expected lying outside the uncertainty range, also the value of $g$ was different to the known value however, the angle of incidence and reflection were shown to be equal to each other. Experimental values determined for the wavelength of the laser light source for the single and double-slit configurations was found to be $652.1 \pm 0.7$ nm and $655.71 \pm 0.03$ nm respectively, using a 0.3mm aperture, a slit width of $a = 0.04$ mm and slit separation of $d = 0.5$ mm: the value of the device is $650 \pm 10$ nm \cite{Web02}. These values compare well with the device value along with other single and double-slit configurations generating wavelengths that compared well with the device value but, all the uncertainties of the experimental values of the wavelength were underestimated. By understanding water waves and electromagnetic waves, hydroelectric power and chemical spectra analysis could be optimised for the benefit of many modern day appliances.
\thispagestyle{empty}
\end{abstract}
\vspace*{\fill}

\vspace{5mm} % takes one free line before rest of text

\newpage

\tableofcontents
\setcounter{page}{1} % starts page numbering from this contents page

\newpage

\section{Introduction}
\label{sec:introduction}

The Earth's weather in the troposphere is a complicated dynamical system with many factors playing a crucial role in its development. Four of the most important are temperature, pressure, wind velocity and humidity. During this project, the relative humidity (RH) of the troposphere was investigated by treating it as constant (homogeneous) in order to obtain a relative humidity values. By obtaining values for the relative humidity many benefits can be obtained. One benefit might be that by knowing the relative humidity this could be used in our day-to-day lives to gain an idea of what the weather will be on a particular day, another is that this could give more accurate rainfall patterns allowing farmers to optimise weather conditions when planting crops. Furthermore, the prediction of heavy rainfall could be predicated {\textemdash} saving lives in the case of serious flooding. Thus, the relative humidity of the troposphere is of crucial interest to weather forecasters and this project was carried out in association with the Met Office to help develop their forecasting techniques.

\vspace{2mm}
\noindent
At present, this is difficult to measure directly and only a small number of aircraft have such devices, for example Aircraft Meteorological Data Relay (AMDAR), on board \cite{Paper01} and these are often expensive which limits further opportunities using this method of measurement. An alternative approach in measuring humidity which has been proposed \cite{Paper01} is to use Automatic Dependent Surveillance Broadcast (ADSB) signals from air planes where many of these aircraft have this equipment on board. Therefore, these broadcasts containing aircraft information such as position, velocity, altitude and so on, can be continually obtained by ground monitoring facilities from many different aircraft. By studying how ADSB radio wave signals are refracted within the Earth's lower atmosphere, this could provide a way of obtaining the water vapour distribution thus give relative humidity values.


\section{Theory}
\label{sec:theory}

The speed of waves, $v$, can be determined by using 

\begin{equation}
\label{eq:wave-speed}
v = \lambda f
\end{equation}

\vspace{2mm}
\noindent
where $\lambda$ is the wavelength and $f$ is the frequency of the waves - the number of waves that pass a given point per second. However, the speed of mechanical waves often depend upon the medium which they travel, in addition to many other factors. For water waves, the speed depends on depth. For shallow depths, this is given by \cite{Book02}

\begin{equation}
\label{eq:water-waves}
v \approx \sqrt{gh}
\end{equation}

\vspace{2mm}
\noindent
where $g$ is the acceleration due to gravity and $h$ is the measured depth of the water used in the ripple tank. For the interested reader, a more detailed analysis of water waves can be found at \cite{Book02}. 

\vspace{2mm}
\noindent
When a wave front approaches at an angle to a boundary reflection occurs. This is when the angle of incidence, $\theta_i$ is equal to the angle of reflection, $\theta_r$ at the boundary

\begin{equation}
\label{eq:reflection}
\theta_i = \theta_r
\end{equation}

\vspace{2mm}
\noindent
\figref{fig:reflection_refraction} describes this pictorially.

\vspace{2mm}
\noindent


%\begin{figure}[h]
%\centering
%\includegraphics[scale=0.4]{"reflection_refraction".pdf}
%\caption{Here is my image}
%\label{fig:reflection_refraction}
%\end{figure}


\vspace{2mm}
\noindent


\begin{equation}
\label{eq:intensity1}
I = I_0 \frac{sin^2{\alpha}}{\alpha^2}
\end{equation}

\vspace{2mm}
\noindent
where $\alpha = ({\pi asin{\theta}}) / {\lambda}$, $\lambda$ is the wavelength of the light and $a$ is the slit width.

\vspace{2mm}
\noindent
For two slits, the intensity of diffracted light into an angle $\theta$ \cite{Paper02} is

\begin{equation}
\label{eq:intensity2}
I = I_0 \frac{sin^2{\alpha}}{\alpha^2} cos^2{\delta}
\end{equation}

\vspace{2mm}
\noindent
where $\delta = ({\pi dsin{\theta}}) / {\lambda}$ and $d$ is the separation distance between the two slits. A derivation for equations \eqref{eq:intensity1} and \eqref{eq:intensity2} involving a geometrical approach and phasor diagrams can be found from \cite{Book01}. These equations can also be dervied by considering Fourier transforms and the convolution theorem and approach using these techniques can be found from \cite{Web01}.

\vspace{2mm}
\noindent
For the experimental configuration that was considered, the small angle approximation could be used. Therefore the single slit intensity equation \eqref{eq:intensity1} can be re-expressed as

\begin{equation}
\label{eq:intensity_small1}
I \approx I_0 \frac{sin^2{\frac{\pi a x}{L \lambda}}}{\big(\frac{\pi a x}{L \lambda}\big)^2}
\end{equation}

\vspace{2mm}
\noindent
where the intensities are normalised so, $I_0 = 1$ and $L$ is the slit to screen separation distance and $x$ is the screen position ($L >> x$).

\vspace{2mm}
\noindent
Similarly, the double slit intensity equation \eqref{eq:intensity2} can be re-expressed due to to the small angle approximation

\begin{equation}
\label{eq:intensity_small2}
I \approx I_0 \frac{sin^2{\frac{\pi a x}{L \lambda}}}{\big(\frac{\pi a x}{L \lambda}\big)^2} cos^2{\frac{\pi d x}{L \lambda}}
\end{equation}

\vspace{2mm}
\noindent
where again the intensities are normalised in the experiments so $I_0 = 1$ and all the other symbols have their usual meaning with $L >> x$.



\vspace{2mm}
\noindent
Using equation \eqref{eq:intensity1}, the intensity minima for a single slit occurs when $\alpha$ is a multiple of $\pi$. This therefore means that

\begin{equation}
\label{eq:min_intensity}
sin{\theta} = \frac{m \lambda}{a} \quad,\quad(m = \pm1, \pm2, ...)
\end{equation}

\vspace{2mm}
\noindent
If L'H{\^o}pital's rule \cite{Book01} is applied to equation \eqref{eq:intensity1} in the limit as the intensity at $\alpha \rightarrow 0$ is found to be $I = I_0$ as expected. When the detector to slit distance is big, the small angle approximation can be used. Equation \eqref{eq:min_intensity} can then be approximated to

\begin{equation}
\label{eq:small_angle}
\theta \approx \frac{m\lambda}{a} \quad,\quad(m = \pm1, \pm2, ...)
\end{equation}


\vspace{2mm}
\noindent
The intensity maxima of a single slit diffraction pattern can be approximately found by using equation \eqref{eq:intensity1} and realising that they occur when the sine function is a maximum ($\pm 1$) \cite{Book01}. In other words, when

\begin{equation}
\label{eq:alpha_max}
\alpha = \pm\bigg(m + \frac{1}{2}\bigg)\pi \quad,\quad(m = 0, 1, 2, ...)
\end{equation}

\vspace{2mm}
\noindent
However, upon further analysis of applying differentiation to equation \eqref{eq:intensity1} and setting equal to zero to find the maxima it is found that the there is no maxima at $m = 0$ \cite{Book01}. Therefore, when equation \eqref{eq:alpha_max}, with $m \neq 0$, is subsituted into equation \eqref{eq:intensity1} the intensity maxima, $I_m$ is approximately given by \cite{Book01}

\vspace{2mm}
\noindent
\begin{equation}
\label{eq:max_intensity}
I_m \approx \frac{I_0}{\big(m + \frac{1}{2}\big)^2\pi^2} \quad,\quad(m =  \pm1, \pm2, ...)
\end{equation}


\section{Method}
\label{sec:method}

\subsection{Mechanical Waves}
\label{ssec:ripple-method}


\subsection{Electromagnetic Waves}
\label{ssec:diffraction-method}


\section{Results}

\subsection{Electromagnetic Waves}
\label{ssec:diffraction-results}


\section{Discussion}
\label{sec:discussion}



\section{Conclusion}
\label{sec:conclusion}




\begin{thebibliography}{9}
\bibitem{Paper01} Stone, E.K and Kitchen M, 2015, Introducing an Approach for Extracting Temperature from Aircraft GNSS, Journal of Atmospherics and Oceanic Technology, Vol 32, pages 736 - 743. 
and Pressure Altitude Reports in ADS-B Messages
\bibitem{Paper02} College of Engineering, Mathematics and Physical Sciences, University of Exeter, PHY2026, \textit{Diffraction and Interference Worksheet} (Accessed 8\textsuperscript{th} February 2019).
\bibitem{Book01} Young, Hugh D and Freedman, Roger A, \textit{University Physics}, 13\textsuperscript{th} Edition, 2014, Chapter 36, pages 1312 - 1322.
% \bibitem{Book02} Young, Hugh D and Freedman, Roger A, \textit{University Physics}, 13\textsuperscript{th} Edition, Chapter 20, pages 728 - 740, 2014.
\bibitem{Book02} Barber N.F, \textit{Water Waves}, 1\textsuperscript{st} Edition, 1969, Chapter 3, pages 36 - 55. 
\bibitem{Book03} Pedrotti F.L, Pedrotti S.J, \textit{Introduction to Optics}, (Pearson International Edition) 3\textsuperscript{rd} Edition, 2006, Chapter 11-1: \textit{Diffraction from a Single Slit}, pages 268 - 273.
\bibitem{Book04} Pedrotti F.L, Pedrotti S.J, \textit{Introduction to Optics},  (Pears onInternational Edition) 3\textsuperscript{rd} Edition, 2006, Chapter 11-5: \textit{Double-Slit Diffraction}, pages 281 - 284.
\bibitem{Web01} \textit{Fraunhofer Diffraction}, \url{http://people.ucalgary.ca/~lvov/471/labs/fraunhofer.pdf} (Accessed 9\textsuperscript{th} February 2019.)
\bibitem{Web02} \textit{Red Diode Laser - Basic Optics - OS-8525A}, 
\url{https://www.pasco.com/prodCatalog/OS/OS-8525_red-diode-laser-basic-optics/index.cfm?fbclid=IwAR3gzuNSAoumEZpwYZQO6tX3j1nLhTtLYklX5U6V6HHw5xyHKdszO1ID00I} (Accessed 15\textsuperscript{th} March 2019).

\end{thebibliography}

\end{document}