% Comments start with % (percent) character and last till the end of the line.
%
% The line below tells TeXworks editor to use pdflatex for compilation
% of this document; remove it if you want to use another engine 
%
%!TEX program = pdflatex
%
% LaTeX2e document starts with \documentclass[options]{<class-name>}
% <class-name> can be one of the standard LaTeX document classes: 
% article, report or book, or some other specialised class.
%
\documentclass{article}
%
% Preamble of LaTeX document is everything before \begin{document}.
% Preamble is used to load extension packages and to set up global 
% parameters and configuration for the entire document.
%
% Extension packages providing additional functionality
\usepackage{amsmath}       % additional math environments
\usepackage{graphicx}      % graphics import from external files 
\usepackage{epstopdf}      % automates .eps to .pdf conversion
% epstopdf package may require --shell-escape option to pdflatex
\usepackage{booktabs}      % table typesetting additions
\usepackage{siunitx}       % number and units formatting
\usepackage{caption}       % customisation of captions
\usepackage{subcaption} 
\usepackage{url}           % format url addresses
\usepackage{abstract}		% allows formatting of abstract
\usepackage[margin=0.5in]{geometry}
\usepackage{listings} % to allow computer code to be typeset


%\usepackage{tikz,pgfplots} % diagrams and data plots
%
% set up caption options
\captionsetup{margin=12pt,font=small,labelfont=bf}
%
%removes abstract title

\renewcommand{\abstractname}{Abstract}
\renewcommand{\abstractnamefont}{\normalfont\Large\bfseries}

%sets abstract margins
\setlength{\absleftindent}{10mm}
\setlength{\absrightindent}{10mm}

%
% global options for siunitx
%\sisetup{seperr,repeatunits=false,per=symbol}
%
% some handy commands for referencing;
% the optional argument overrides the default label, e.g.
% \figref[FIG.~]{fig:label}
\newcommand{\figref}[2][\figurename~]{#1\ref{#2}}
\newcommand{\tabref}[2][\tablename~]{#1\ref{#2}}
\newcommand{\secref}[2][Section~]{#1\ref{#2}}


% The document content starts with \begin{document} 
% and is finished with \end{document}
%

\title{\textbf{\huge Predicting the Weather by Watching Aeroplanes: 
\\ 
\Large
Applying Refractive Techniques to Aeroplane ADSB Radio Wave Signals in the Lower Troposphere for Determining Relative Humidities.}}
 % fill in the title here
\author{Matthew Evans}% fill in your name here
\date{27\textsuperscript{th} March 2019} % date of the report
\usepackage{titling} % centralises the title
%\renewcommand\maketitlehooka{\null\mbox{}\vfill}
%\renewcommand\maketitlehookd{\vfill\null}
\thispagestyle{empty}

\begin{document}

% \twocolumn % makes title and abstract appear over entire page width
\maketitle % formats the title

\begin{figure}[h]
\centering
\includegraphics[scale=0.5]{"raincloud".pdf}
\label{fig:raincloud}
\end{figure}

\thispagestyle{empty}
\newpage



\vspace*{\fill}
\begin{abstract}

\noindent
Various wave properties were investigated by exploring water waves in a ripple tank and studying the diffraction of laser light. A value of $ 0.18 \pm 0.05$ ms$^{-1}$ for water wave speed at varying frequencies and a fixed depth of $h = 1.0 \pm 0.3$ cm was obtained and the one generated by using $v \approx \sqrt{gh}$ \cite{Book02} was $ 0.3 \pm 0.1$ ms$^{-1}$. The angles of incidence and reflection for the reflection of water waves at a barrier were found to be $41 \pm 2^{\circ}$ and $41 \pm 2^{\circ}$ respectively. The water wave speed was also investigated at a fixed frequency of $20$ Hz with varying depths allowing a value of $4.4 \pm 0.1$ ms$^{-2}$ for the acceleration due to gravity to be determined with the known value as 9.8 ms$^{-2}$. The value for the wave speed at varying frequencies was different to the one expected lying outside the uncertainty range, also the value of $g$ was different to the known value however, the angle of incidence and reflection were shown to be equal to each other. Experimental values determined for the wavelength of the laser light source for the single and double-slit configurations was found to be $652.1 \pm 0.7$ nm and $655.71 \pm 0.03$ nm respectively, using a 0.3mm aperture, a slit width of $a = 0.04$ mm and slit separation of $d = 0.5$ mm: the value of the device is $650 \pm 10$ nm \cite{Web02}. These values compare well with the device value along with other single and double-slit configurations generating wavelengths that compared well with the device value but, all the uncertainties of the experimental values of the wavelength were underestimated. By understanding water waves and electromagnetic waves, hydroelectric power and chemical spectra analysis could be optimised for the benefit of many modern day appliances.
\thispagestyle{empty}
\end{abstract}
\vspace*{\fill}

\vspace{5mm} % takes one free line before rest of text

\newpage

\tableofcontents


\setcounter{page}{1} % starts page numbering from this contents page

\newpage


\section{Introduction}
\label{sec:introduction}

The Earth's weather in the troposphere is a complicated dynamical system with many factors playing a crucial role in its development. Four of the most important are temperature, pressure, wind velocity and humidity. During this project, the relative humidity (RH) of the troposphere was investigated by treating it as constant (homogeneous) in order to obtain a relative humidity values. By obtaining values for the relative humidity many benefits can be obtained. One benefit might be that by knowing the relative humidity this could be used in our day-to-day lives to gain an idea of what the weather will be on a particular day, another is that this could give more accurate rainfall patterns allowing farmers to optimise weather conditions when planting crops. Furthermore, the prediction of heavy rainfall could be predicated {\textemdash} saving lives in the case of serious flooding. Thus, the relative humidity of the troposphere is of crucial interest to weather forecasters and this project was carried out in association with the Met Office to help develop their forecasting techniques.

\vspace{2mm}
\noindent
At present, this is difficult to measure directly and only a small number of aircraft have such devices, for example Aircraft Meteorological Data Relay (AMDAR), on board \cite{Paper01} and these are often expensive which limits further opportunities using this method of measurement. An alternative approach in measuring humidity which has been proposed \cite{Paper01} is to use Automatic Dependent Surveillance Broadcast (ADSB) signals from air planes where many of these aircraft have this equipment on board. Therefore, these broadcasts containing aircraft information such as position, velocity, altitude and so on, can be continually obtained by ground monitoring facilities from many different aircraft. By studying how ADSB radio wave signals are refracted within the Earth's lower atmosphere, this could provide a way of obtaining the water vapour distribution thus give relative humidity values.


\section{Theory}
\label{sec:theory}

The refractive index, $n$ is defined by

\begin{equation}
\label{eq:ref_index}
n = \frac{c}{v}
\end{equation}

\vspace{2mm}
\noindent
where $c$ is the speed of an electromagnetic (EM) wave in vacuum and $v$ is the phase velocity in a medium. An important relation that arises from the refractive is Snell's Law. This states that the refractive index multiplied by the sine of the angle between the incident EM wave and the normal to the interface surface between two different media of refractive indices $n_1$ and $n_2$ is conserved:

\begin{equation}
\label{eq:Snell_law}
n_1\sin{\theta_1} = n_2\sin{\theta_2}
\end{equation}

\vspace{2mm}
\noindent
$\theta_1$ is the angle between the incident ray and the normal (incident angle), $\theta_2$ is the angle between the transmitted ray and the normal (refracted angle). A different form of Snell's Law \eqref{eq:Snell_law} takes into account the radial length of the ray path and is derived as follows: if a radially stratified atmosphere is now considered, \figref{fig:atmosphere} \cite{Paper02}, Snell's Law \eqref{eq:Snell_law} can be applied by considering these atmospheric layers as different media with different respective refractive indices.

\begin{figure}[h]
\centering
\includegraphics[scale=0.67]{"atmosphere".pdf}
\caption{A visual illustration the path of a refracted EM ray as it propagates through a radially startified atmosphere. Image adapted from \cite{Paper02}.}
\label{fig:atmosphere}
\end{figure}

\vspace{2mm}
\noindent
Applying equation \eqref{eq:Snell_law} to the upper two layers in \figref{fig:atmosphere} with refractive indices $n_N$ and $n_{N-1}$ the following is obtained

\begin{equation}
\label{eq:Snell_law2}
n_N\sin{\phi_N} = n_{N-1}\sin{\psi}
\end{equation}

\vspace{2mm}
\noindent
where $\phi_N$ and $\psi$ are the incident and refracted angles respectively. By inspecting the triangle CQP in \figref{fig:atmosphere} and using the law of sines, also noticing that $\sin{\angle{CQP}} \equiv \sin{\phi_{N-1}}$, 

\begin{equation}
\label{eq:triangle}
n_Nr_N\sin{\phi_N} = n_{N-1}r_{N-1}\sin{\phi_{N-1}}
\end{equation}


\vspace{2mm}
\noindent
where $phi_N$ and $phi_{N-1}$ are the local zenith angles in \figref{fig:atmosphere} with refractive indices $n_N$ and $n_{N-1}$ respectively, $r_N \equiv CP$ and $r_{N-1} \equiv CQ$ are the radial distance defined in \figref{fig:atmosphere}. Equation \eqref{eq:triangle} can be rearranged to yield the radial distance $r_N$ the subject

\begin{equation}
\label{eq:Snell_law3}
r_N = \frac{n_{N-1}r_{N-1}\sin{\phi_{N-1}}}{n_N\sin{\phi_N}}
\end{equation}

\vspace{2mm}
\noindent
where the radial distance $r_N$ is now the subject of the equation.

\vspace{2mm}
\noindent
The refractive index can vary in value from $n = 1$ for a vacuum to higher values for other media. Typically, more water vapour present in the air will cause the refractive index to increase because the phase velocity of an EM wave given in \eqref{eq:ref_index} will decrease. This leads to a useful alternative definition for the refractive index, $n$ in terms of the refractivity, $N$ given in parts per million (ppm)

\begin{equation}
n = 1  + (N \times 10^{-6})
\label{eq:refractivity}
\end{equation}

\vspace{2mm}
\noindent
where $n$ is expressed as the ppm excess above the value of unity. A formal definition of the relative humidity, $RH$ \cite{Book01} is given by

\begin{equation}
RH = \frac{P_w}{P_s} \times 100\%
\label{eq:humidity}
\end{equation}

\vspace{2mm}
\noindent
where $P_w$ is the actual/partial water vapour pressure and $P_s$ is the ``saturation water vapour pressure that is possible for a given temperature before condensation occurs'' \cite{Book01}. The refractivity of air is comprised of two parts: a dry component, $N_{dry}$ and a wet component, $N_{wet}$.

\begin{equation}
N = N_{dry} + N_{wet}
\label{eq:refractivity2}
\end{equation}

\vspace{2mm}
\noindent
By using further information from \cite{Paper02} equation \eqref{eq:refractivity2} yields

\begin{equation}
N = \bigg(77.6\frac{P}{T} - 5.6\frac{P_w}{T} + 3.75 \times 10^5\frac{P_w}{T^2} \bigg) \times 10^6
\label{eq:refractivity3}
\end{equation}

\vspace{2mm}
\noindent
where $P$ is the total pressure (in hPa), $T$ is the ambient air temperature (in K) and, as before in the definition of the refractivity \eqref{eq:refractivity}, $N$ is given in ppm. Expressions for $P$, $T$ and $P_w$ in equation \eqref{eq:refractivity3} will now be outlined. Further information and derivations regarding these expressions can be found in \cite{Paper03} (for $P$ and $T$) and \cite{Paper04} (for $P_w$). It can be shown that $T$ varies as a function of altitude given by the following expression

\begin{equation}
T = T_b + \beta(H - H_b)
\label{eq:temperature}
\end{equation}

\vspace{2mm}
\noindent
where $T_b$ is the baseline temperature, $H$ is the height above the baseline height $H_b$, both measure in km and $\beta$ is the temperature gradient (Kkm$^{-1}$). In this case, the troposphere was considered thus, using the information from \cite{Paper03} $\beta = - 6.5$Kkm$^{-1}$. An expression for $P$ is given as,

\begin{equation}
P = P_b \bigg( 1 + \frac{\beta}{T_b}(H - H_b) \bigg) ^{-\frac{gM_0}{\beta R}}
\label{eq:pressure}
\end{equation}

\vspace{2mm}
\noindent
where $P_b$ is the baseline pressure, $g$ is the acceleration due to gravity (in ms$^2$), $M_0$ is the molar mass of air (in kgmol$^{-1}$) and $R$ is the gas constant (in JK$^{-1}$mol$^{-1}$). The partial pressure water vapour pressure $P_w$ can be expressed as the Arden-Buck equation, which is adapted from \cite{Paper04}

\begin{equation}
P_w = \frac{RH}{100} 6.1121exp \bigg( \bigg(18.678 - \frac{T'}{234.5} \bigg) \bigg (\frac{T'}{257.14 + T'} \bigg) \bigg)
\label{eq:Arden-Buck}
\end{equation}

\vspace{2mm}
\noindent
where $T'$ is the temperature in \textit{Celcius} ($^{\circ}$).

\section{Method}
\label{sec:method}

\subsection{Radio Interferometry}
\label{ssec:radio-interferometry}

The refractive techniques discussed in \secref{sec:theory} can be applied to ADSB signals transmitted from aircraft. These are 1090 MHz radio waves which are 112 bits long \cite{Book02}, containing aircraft information such as position, velocity, altitude, etc. Therefore, these will be refracted as an EM wave passing through the atmosphere. These broadcasts can be received by using radio interferometric techniques. There are two radio interferometers placed on the fourth and seventh floors of the Physics Building at the University of Exeter, this is illustrated in \figref{fig:interferometry}.

\begin{figure}[h]
\centering
\includegraphics[scale=0.5]{"interferometry".pdf}
\caption{A diagram illustrating interferometry used to receive ADSB signals. $t$ and $t - \Delta t$ are the times when the first and second radio interferometers, respectively received the ADSB signals, $b$ is the baseline distance between the two receivers, $V_1$ and $V_2$ are the voltages of the first and second signals respectively, $\phi_0$ is the observed angle, $z$ is the zenith angle and $b \cos z$ is the path difference between the two signals. The value of $b$ is known from internal project communications and $\lambda$ is the wavelength of the ADSB signal.}
\label{fig:interferometry}
\end{figure}

\vspace{2mm}
\noindent
By taking the conjugate product, $V_1 V_2^*$, the time average voltage, $V^2/2$ and the phase, $\alpha = \omega \Delta t = (2 \pi b / \lambda) \cos z$, can be obtained to collect all the relevant aircraft information required for further analysis.


\subsection{Data Analysis}
\label{ssec:data-analysis}

In order to obtain constant, homogeneous relative humidity values, the first step was to use equation \eqref{eq:Snell_law3} in a computer program, provided by internal project communications, that calculates the radial path of the refracted ray (see Appendix A). This was achieved by using equations \eqref{eq:temperature}, \eqref{eq:pressure} and \eqref{eq:Arden-Buck} in equation \eqref{eq:refractivity3} and modifying the program to obtain a more accurate refractivity value, $N$ at each iteration of the propagating wave. This would then generate model 0$\%$ - 100$\%$ RH curves via inputting initial conditions: the observed angle above the horizon, $\phi_0$ and increment angle step from the vertical, $\delta \theta$. These models can then be compared to the plane data positions, given in radial distance from the centre of the Earth, $R$ (km) to the signal position and the increment angle, $\theta$ (rad), measured from the vertical through the centre of the Earth (where $R$ and $\theta$ are illustrated in \secref{sec:theory} \figref{fig:atmosphere}). 

\vspace{2mm}
\noindent
The chosen data set for analysis was 26\textsuperscript{nd} August 2019 starting at 11:45am because this had a nice complete set of data that could be used for further data analysis and was recommended by internal project communications. This data set was then broken-up into early, first 5000s, and late, last 5000s, times of observation as it was anticipated that during the course of the day $T_b$ and $P_b$ would change. These baseline values were obtained from \cite{Web01} (using nearby locations in Exeter) and an average for temperature, $\overline{T_b}$ and pressure, $\overline{P_b}$ was taken for the corresponding times to input into the computer program. Three observed angle ranges were considered: 0.1$^{\circ}$ - 0.11$^{\circ}$, 0.3$^{\circ}$ - 0.31$^{\circ}$ and 0.5$^{\circ}$ - 0.51$^{\circ}$. This gave a good idea of the behaviour of relative humidities at different altitudes. The average of these observed angle ranges, $\overline{\phi_0}$ was then taken from the sub-sectioned data sets in order to feed into the computer program generating the corresponding model curves.

\vspace{2mm}
\noindent
Once these model curves had been generated, the data points (plane positions) would be interpolated onto each of the model relative humidity curves using another computer program (see Appendix B). Finally, a third computer program obtains the least-squares difference to find the best fitting model curve that would be represent the plane position data outputting the corresponding relative humidity value in the process (see Appendix C).

\vspace{2mm}
\noindent
The generated RH values would then be compared to the actual, average, RH value, $\overline{RH}$ on this day for the corresponding early and late times given by \cite{Web01}.


\section{Results}
\label{sec:results}
\figref{fig:phi0_0.105} shows two plots of $R$ against $\theta$ for a given observed angle range of 0.1$^{\circ}$ - 0.11$^{\circ}$ for early (\figref{fig:sub1}) and late (\figref{fig:sub2}) times of observation. Furthermore, \figref{fig:phi0_0.305} and \figref{fig:phi0_0.505} show additional plots of these $R$ and $\theta$ positions for different observed angle ranges of 0.3$^{\circ}$ - 0.31$^{\circ}$ and 0.5$^{\circ}$ - 0.51$^{\circ}$ respectively. Again, the early times of observation are shown for these different observed angles ranges in \figref{fig:sub3} and \figref{fig:sub5} at observed angle ranges of 0.3$^{\circ}$ - 0.31$^{\circ}$ and 0.5$^{\circ}$ - 0.51$^{\circ}$ respectively. The late times are shown in \figref{fig:sub4} and \figref{fig:sub6} for the observed angle ranges of 0.3$^{\circ}$ - 0.31$^{\circ}$ and 0.5$^{\circ}$ - 0.51$^{\circ}$ respectively. All these plot values are shown for the plane positions, in addition to the model relative humidity curves for 0$\%$, 100$\%$ and the best fitting relative humidity curve for the plane positions.


\begin{figure}[h]
\centering
\begin{subfigure}{.5\textwidth}
  \centering
  \includegraphics[scale=0.64]{"aug26_0_105_early".pdf}
  \caption{$R$ vs $\theta$ for early times (the first 5000s) of observation.}
  \label{fig:sub1}
\end{subfigure}%
\begin{subfigure}{.5\textwidth}
  \centering
  \includegraphics[scale=0.64]{"aug26_0_105_late".pdf}
  \caption{$R$ vs $\theta$ for late times (the last 5000s) of observation.}
  \label{fig:sub2}
\end{subfigure}
\caption{Plots of radial distance, $R$ against increment angle, $\theta$ from the vertical through the centre of the Earth for plane positions (shown in black) at a given observed angle range between 0.1$^{\circ}$ - 0.11$^{\circ}$ for early and late times of observation.}
\label{fig:phi0_0.105}
\end{figure}

\newpage
\begin{figure}[h]
\centering
\begin{subfigure}{.5\textwidth}
  \centering
  \includegraphics[scale=0.64]{"aug26_0_305_early".pdf}
  \caption{$R$ vs $\theta$ for early times (the first 5000s) of observation.}
  \label{fig:sub3}
\end{subfigure}%
\begin{subfigure}{.5\textwidth}
  \centering
  \includegraphics[scale=0.64]{"aug26_0_305_late".pdf}
  \caption{$R$ vs $\theta$ for late times (the last 5000s) of observation.}
  \label{fig:sub4}
\end{subfigure}
\caption{Plots of radial distance, $R$ against increment angle, $\theta$ from the vertical through the centre of the Earth for plane positions (shown in black) at a given observed angle range between 0.3$^{\circ}$ - 0.31$^{\circ}$ for early and late times of observation.}
\label{fig:phi0_0.305}
\end{figure}


\begin{figure}[h]
\centering
\begin{subfigure}{.5\textwidth}
  \centering
  \includegraphics[scale=0.64]{"aug26_0_505_early".pdf}
  \caption{$R$ vs $\theta$ for early times (the first 5000s) of observation.}
  \label{fig:sub5}
\end{subfigure}%
\begin{subfigure}{.5\textwidth}
  \centering
  \includegraphics[scale=0.64]{"aug26_0_505_late".pdf}
  \caption{$R$ vs $\theta$ for late times (the last 5000s) of observation.}
  \label{fig:sub6}
\end{subfigure}
\caption{Plots of radial distance, $R$ against increment angle, $\theta$ from the vertical through the centre of the Earth for plane positions (shown in black) at a given observed angle range between 0.5$^{\circ}$ - 0.51$^{\circ}$ for early and late times of observation.}
\label{fig:phi0_0.505}
\end{figure}


\vspace{2mm}
\noindent
\tabref{tab:table1} compares the obtained values for the relative humidity from the best fitting (green) curves in \figref{fig:sub1}, \figref{fig:sub3} and \figref{fig:sub5} to the mean of the actual relative humidities obtained from \cite{Web01} along with their estimated errors for each of the observed angle ranges at early times of observation.

\vspace{5mm}
\begin{table*}[h]
\centering % centre table
\caption{A table comparing generated relative humidities (RH) from an interpolative and least squares method for the various different observed angle, $\phi_0$ ranges investigated to the mean of the actual RH values given by \cite{Web01} for the early times of observation (first 5000s).}
\label{tab:table1}
\begin{tabular}{|c|c|c|}
\hline
Observed Angle Range $\phi_0$ ($^{\circ}$) & Generated RH ($\%$) & Actual RH ($\%$) \cite{Web01} \\
\hline
0.10 - 0.11 & $99 \pm 8$ & $74 \pm 4$ \\
\hline
0.30 - 0.31 & $48 \pm 3$ & $74 \pm 4$ \\
\hline
0.50 - 0.51 & $38 \pm 3$ & $74 \pm 4$ \\
\hline
\end{tabular}
%\end{center}
\end{table*}

\vspace{2mm}
\noindent
\tabref{tab:table2} compares the obtained values for the relative humidity from the best fitting (green) curves in \figref{fig:sub2}, \figref{fig:sub4} and \figref{fig:sub6} to the mean of the actual relative humidities obtained from \cite{Web01} along with their estimated errors for each of the observed angle ranges at late times of observation.

\vspace{5mm}
\begin{table*}[h]
\centering % centre table
\caption{A table comparing generated relative humidities (RH) from an interpolative and least squares method for the various different observed angle, $\phi_0$ ranges investigated to the mean of the actual RH values given by \cite{Web01} for the late times of observation (last 5000s).}
\label{tab:table2}
\begin{tabular}{|c|c|c|}
\hline
Observed Angle Range $\phi_0$ ($^{\circ}$) & Generated RH ($\%$) & Actual RH ($\%$) \cite{Web01} \\
\hline
0.10 - 0.11 & - & $77 \pm 5$ \\
\hline
0.30 - 0.31 & 11 $\pm$ $<$1 & $77 \pm 5$ \\
\hline
0.50 - 0.51 & - & $77 \pm 5$ \\
\hline
\end{tabular}
%\end{center}
\end{table*}


\vspace{15mm}
\noindent
The errors for the actual relative humidities for the early and late times of observation, given by \cite{Web01}, were estimated by using the error on the mean value of the relative humidity $\delta{\overline{RH}}$,


\begin{equation}
\label{eq:RH_error}
\delta{\overline{RH}} = \frac{\sigma{(RH)}}{\sqrt{N_{RH}}}
\end{equation}

\vspace{2mm}
\noindent
where $\sigma{(RH)}$ is the standard deviation of the average actual relative humidity ($\overline{RH}$) and $N_{RH}$ is the number of RH values used to determine these averages.

\vspace{2mm}
\noindent
However, the errors for the generated relative humidities were estimated in \tabref{tab:table1} and \tabref{tab:table2} by using the error on the mean for the baseline temperature $\overline{T_b}$ \cite{Web01}, baseline pressure $\overline{P_b}$ \cite{Web01} and observed angle $\overline{\phi_0}$. The following were used to obtain these errors

\begin{equation}
\label{eq:temp_error}
\delta{\overline{T_b}} = \frac{\sigma{(T_b)}}{\sqrt{N_{T{_b}}}}
\end{equation}


\begin{equation}
\label{eq:press_error}
\delta{\overline{P_b}} = \frac{\sigma{(P_b)}}{\sqrt{N_{P{_b}}}}
\end{equation}

\begin{equation}
\label{eq:angle_error}
\delta{\overline{\phi_0}} = \frac{0.03}{\sqrt{N_{\phi{_0}}}}
\end{equation}

\vspace{2mm}
\noindent
where $\sigma{(T_b)}$, $\sigma{(P_B)}$ are the standard deviations of the baseline temperature and pressure respectively, 0.03 is the known error in the observed angle as given by internal project communications, $N_{T{_b}}$, $N_{P{_b}}$ and $N_{\phi{_0}}$ are the number of baseline temperature, pressure and observed angle readings considered. 

\vspace{2mm}
\noindent
These estimated errors given in \eqref{eq:temp_error}, \eqref{eq:press_error} and \eqref{eq:angle_error} were added to the mean values for the baseline temperature, baseline pressure and observed angle when calculating the refracted ray path using the appropriate computer program. This gave an upper bound on the relative humidity value. Thus, the difference between the upper bound of the relative humidity and the relative humidity generated by using the average of these values gave an estimated error for the generated relative humidity value.

\section{Discussion}
\label{sec:discussion}

The results for the determined relative humidities given for the early and late part of the observations in \tabref{tab:table1} and \tabref{tab:table2} respectively show that the relative humidity does vary as a function of time. This is due to the fact that later in the day, the temperature and pressure decreases causing a change in the relative humidity as anticipated in \secref{sec:theory}.

\vspace{2mm}
\noindent
By considering the early part of the day, the relative humidities given in \tabref{tab:table1} show discrepencies when compared with the actual relative humidities. The corresponding graphs displaying these best fitting relative humidity curves in \figref{fig:sub1}, \figref{fig:sub3} and \figref{fig:sub5} visually show these discrepancies. For example, in \figref{fig:sub3} and \figref{fig:sub5} show that indeed the plane positions are grouped towards a lower relative humidity. However, when compared to the actual values displayed in \tabref{tab:table1} these generated values are much lower. This could be due to the fact that the relative humidity cannot be treated as a constant, homogeneous value. Instead, the relative humidity could be varying as a function of many different factors; such as temperature, pressure, height, etc. 

\vspace{2mm}
\noindent
Furthermore, by considering \figref{fig:sub1} the plane positions seem to be forming two distinct relative humidity curves. This shows that the relative humidity does indeed seem to be different for a specific observed angle range. In this case, there is a shallow observed angle range 0.1$^{\circ}$ - 0.11$^{\circ}$ above the horizon. From the discussion presented in \secref{sec:theory} \eqref{eq:refractivity3} this is where most of the refraction occurs due to the denser atmosphere nearer the Earth's surface. Therefore, it would be expected that more refraction would occur because the ADSB signal spends more time in the lower part of the atmosphere for these small observed angles above the horizon according to \secref{sec:theory} \figref{fig:atmosphere}. This is shown by the bottom set of plane position data values in \figref{fig:sub1}. However, there is also a set of points that indicate a higher relative humidity. This was investigated further and additional measurements, using the latitude and longitude plane positions, show that the relative humidity, as well as depending on time, also depends on upon the horizontal direction (azimuthal angle). \figref{fig:long_lat_plot} shows a plot of latitude against longitude for the observed angle range 0.1$^{\circ}$ - 0.11$^{\circ}$ for early observation times.


\begin{figure}[h]
\centering
\includegraphics[scale=0.65]{"aug26_0_105_long_lat_early".pdf}
\caption{A plot showing the latitude and longitude plane positions during the early part of observations (first 5000s) for the observed angle range 0.1$^{\circ}$ - 0.11$^{\circ}$.}
\label{fig:long_lat_plot}
\end{figure}


\vspace{2mm}
\noindent
\figref{fig:long_lat_plot} shows that for this given observed angle range there are three distinct horizontal regions for these plane positions. This shows that the relative humidity also depends upon the horizontal orientation as well as time.

\vspace{2mm}
\noindent
Analysing the results for the determined relative humidities given for the late part of the observations in \tabref{tab:table2} show further discrepancies when compared to the actual relative humidities. Two of these relative humidities could not be determined due to the nature of the graphs displayed in \figref{fig:sub2}, and \figref{fig:sub6}. These plots show that the plane positions would yield a best fit relative humidity curve that would be less than 0$\%$. This further indicates that the relative humidity cannot be treated as an absolute constant value and that the gradient of the relative humidity curves in addition to how they curves vary with height should be considered. 


\vspace{2mm}
\noindent
By considering the refractivity as a function of altitude, this would mean that a higher ground level relative humidity would give a steeper fall-off gradient with height due to the refractivity relations given in \secref{sec:theory}. Indeed by analysing the corresponding refractivity curve against height for \figref{fig:sub2} the following plot is generated in \figref{fig:refractivity_plot}.

\begin{figure}[h]
\centering
\includegraphics[scale=0.65]{"0_105_refractivity_plot".pdf}
\caption{A plot showing the refractivity $N$ as a function of altitude $h$ (km) for 0$\%$ and 100$\%$ relative humidities at the observed angle range 0.1$^{\circ}$ - 0.11$^{\circ}$ for late times of observation (last 5000s).}
\label{fig:refractivity_plot}
\end{figure}

\newpage
\vspace{2mm}
\noindent
So, the plots \figref{fig:sub2}, \figref{fig:sub4} and \figref{fig:sub6} show that the generated relative humidities are less than 0$\%$. According to \figref{fig:refractivity_plot} this would mean a negative absolute value of relative humidity with a shallower gradient refractivity curve. Therefore, a systematic shift would be required for the plane positions, so that a model refractivity curve with a low value of relative humidity could be used to represent the data. This would display the same shallow gradient characteristics of the plane position data for low relative humidities, but translated to a higher refractivity/relative humidity. This change in relative humidity with height above ground level for the translated curve, would result in an initially rising relative humidity with altitude. However, this would have to fall off after some height as this translated refractivity curve, with a shallow gradient, would intersect the 100$\%$ refractivity curve after some altitude (which has a steeper fall off gradient) according to \figref{fig:refractivity_plot} - the translated curve would not be able to go above 100$\%$ due to the definition of relative humidity given in \secref{sec:theory} equation \eqref{eq:humidity}. Therefore, this means that the relative humidity would initially rise and then fall off again with altitude when this intersection between the two refractivity curves occurs. Furthermore, this analysis indicates that the absolute value of the relative humidity at ground level may not be the ultimate factor, but the gradient (i.e. how the humidity behaves as a function of height) could play a more vital role when determining relative humidities.


\section{Conclusion}
\label{sec:conclusion}

Overall, the results presented in \secref{sec:results} show some interesting properties of relative humidity that were previously not well known. For example, as well as depending on time and vertical elevation angle, the relative humidity also depends on the horizontal orientation. Thus, by looking in a different direction, different weather systems could be observed at the same as displayed in \secref{sec:results} \figref{fig:sub1}. Although the generated relative humidities presented in \secref{sec:results} \tabref{tab:table1} and \tabref{tab:table2} were not in conjunction with the actual values from \cite{Web01}, it was shown that the absolute value may not be the only way of describing these values. By analysing the corresponding refractivity curves, such as \figref{fig:refractivity_plot} in \secref{sec:discussion} for \figref{fig:sub2}, it was indicated that the gradient of the $R$ vs $\theta$ curves should be taken into account. Therefore, instead of the relative humidity been treated as a constant, homogeneous value, this should be treated as a variable factor shares a relation with time, horizontal orientation and altitude height. Ultimately, this would give more accurate and reliable relative humidity values; thus enabling better precision for the prediction of the weather by forecasting companies such as the Met Office.


\begin{thebibliography}{9}
\addcontentsline{toc}{section}{References}
\bibitem{Paper01} Stone E.K and Kitchen M, 2015, \textit{Introducing an Approach for Extracting Temperature from Aircraft GNSS}, Journal of Atmospherics and Oceanic Technology, Vol 32, pages 736 - 743. 
and Pressure Altitude Reports in ADS-B Messages
\bibitem{Paper02} Mangum J and Wallace P, 2015, \textit{Atmospheric Refractive Electromagnetic Wave Bending and Propagation Delay}, Publications of the Astronomical Society of the Pacific, Vol.127, pages 74 - 91.
\bibitem{Book01} Vallis Geoffrey K, 2017, \textit{Atmospheric and Oceanic Fluid Dynamics (Fundamentals and Large-Scale Circulation}, 2\textsuperscript{nd} Edition, Chapter 18.1.2, pages 675 - 676.
\bibitem{Paper03} 1993, \textit{Manual of the ICAO Standard Atmosphere}, Third Edition, pages 7 - 14.
\bibitem{Paper04} Buck A.L, 1981, \textit{New Equations for Computing Vapor Pressure and Enhancement Factor}, National Center for Atmospheric Research, Vol.20, pages 1527 - 1532.
\bibitem{Book02} Sun J, \textit{The 1090MHz Riddle: Decoding Mode-S and ADS-B Data}, 29\textsuperscript{th} September 2019 , \url{https://mode-s.org/decode/}.  
\bibitem{Web01} Met Office Weather Observations Website, 22\textsuperscript{nd} November 2019,  \url{https://wow.metoffice.gov.uk/}.


\end{thebibliography}

\appendix
\section*{Appendix A: Code for Calculating the Path of a Refracted Ray}
\label{app:appendix_A}
\addcontentsline{toc}{section}{Appendix A: Code for Calculating the Path of a Refracted Ray}

\begin{footnotesize}
\begin{lstlisting}[language=c]

#include <stdio.h>
#include <stdlib.h>
#include <math.h>
#include <string.h>
#include "nrutil.h"

main(argc,argv)
int argc;
char **argv;
{

   int ndata;
   char curr;
   int i, j;
   double pi;
   double *r,*rray,*theta,*phi,*phib,*n,*ndry,*nwet,*ninf;
   double *T, *P, *init_P; // creating arrays for temp, and pressure
   double *val_1, *val_2, *P_w; // creating arrays for the exponent values for the wet pressure contribution
   double input_phi0, phi0,r0;
   double input_dtheta, dtheta;
   double RH;
   double beta, Pb, Tb; // declaring the temperature gradient, sea-level pressure and temp respectively
   double grav, R, M_0; // declaring the acceleration due to gravity, the gas constant and molar mass of air 
   double logscale;
   int ntheta;
   FILE *fp,*gp;
   char infile[80],outfile[80];

   FILE *files[101]; // a file pointer, pointing to 101 RH data files for a given observed angle

   if (argc!=3)
      { fprintf(stderr,"\n Usage: refractray phi0(deg) dtheta(rad)\n\n");
      exit(1);
   }
	 
   /* sscanf(*++argv,"%s",infile);*/  /* input file name */
   // sscanf(*++argv,"%s",outfile);  /* output file name */
   sscanf(*++argv,"%lf",&input_phi0);    /* initial co-zenith angle */
   sscanf(*++argv,"%lf",&input_dtheta);    /* increment in theta */
   // sscanf(*++argv,"%lf",&RH); /* relative humidity in percent */

   for(int i = 0; i < 101; i++) {
      RH = i;
      phi0 = input_phi0;
      dtheta = input_dtheta;

      sprintf(outfile, "%.3fphi0_%03dRH.dat", phi0, i); /* output file name */
      files[i] = fopen(outfile, "w"); /* list of output files with output file name given above */


      pi=acos(-1.0);
      phi0=pi/2.0-phi0*pi/180.0;
      beta = -6.5; // temp gradient in K(km^-1)
      Pb = 1018; // sea-level pressure in hPa (1013.25)
      Tb = 293.15; // sea-level temp in K (288.15)
      grav = 9.80665;
      R = 8.31432;
      M_0 = 0.028964420; // in kg(mol)^-1
      r0=6365.5; 

      ntheta=(int)5.0*pi/180.0/dtheta;


      phi=dvector(0,ntheta-1);
      phib=dvector(0,ntheta-1);
      rray=dvector(0,ntheta-1);
      n=dvector(0,ntheta-1);
      theta=dvector(0,ntheta-1);
     
      T = dvector(0,ntheta-1); // initialising an array for temperatures
      P = dvector(0,ntheta-1); // intitalising an array for pressures
      init_P = dvector(0,ntheta-1);
      val_1 = dvector(0,ntheta-1); 
      val_2 = dvector(0,ntheta-1);
      P_w = dvector(0,ntheta-1);
     
      val_1[0] = 18.678 - ((Tb - 273.15) / 234.5);
      val_2[0] = (Tb - 273.15) / (Tb - 16.01);
      P_w[0] = (RH / 100) * 6.1121 * exp(val_1[0] * val_2[0]);

      theta[0]=0.0;
      rray[0]=r0;
      phi[0]=phi0;
      // printf("First phi value is: %f\n", phi[0]);
      T[0] = Tb;
      n[0]=1.0 + (77.6*Pb)/Tb/1e6 - (5.6*P_w[0])/Tb/1e6 + (3.75e5*P_w[0])/pow(Tb, 2)/1e6;
      phib[0]=pi/2.0-phi0;
   
      // printf("phi angle is phi[%f]\n", phi[i]);
      for (j=1;j<=ntheta;j++)
         {
            
            theta[j]=dtheta*j;
            // printf("theta angle is theta[%f]\n", theta[j]);
            rray[j]=rray[j-1]*sin(phi[j-1]-theta[j-1])/sin(phi[j-1]-theta[j]);

            T[j] = (beta * (rray[j] - rray[0])) + Tb;
            init_P[j] = (((beta * (rray[j] - rray[0])) + Tb)) / Tb; // (in hPa)
            double exp_P =  (-grav * M_0 * 1000) / (beta * R);
            P[j] = Pb * pow(init_P[j], exp_P);
            
            val_1[j] = 18.678 - ((T[j] - 273.15) / 234.5);
            val_2[j] = (T[j] - 273.15) / (T[j] - 16.01);
            P_w[j] = (RH / 100) * 6.1121 * exp(val_1[j] * val_2[j]);

            n[j] = 1 + (77.6 * P[j]/T[j])/1e6 - (5.6 * P_w[j])/T[j]/1e6 + (3.75e5 * P_w[j])/pow(T[j], 2)/1e6;

            phi[j]=asin(n[0]*rray[0]*sin(phi[0])/n[j]/rray[j])+theta[j];
            phib[j]=atan((rray[j]*cos(theta[j])-rray[0])/(rray[j]*sin(theta[j]))); /* bhatdotrhatv */       
            }

      gp=fopen(outfile,"w");
      // fprintf(gp,"#theta R n z T P_w\n");
      for (j=0;j<ntheta;j++)
         {
          fprintf(gp,"%.16g %.16g %.16g %.16g %.16g %.16g\n",theta[j],rray[j],n[j],phi[j]-theta[j],T[j],P_w[j]);
         }

      free_dvector(theta,0,ntheta-1);
      free_dvector(rray,0,ntheta-1);
      free_dvector(phi,0,ntheta-1);
      free_dvector(phib,0,ntheta-1);
      free_dvector(n,0,ntheta-1);

      free_dvector(T,0,ntheta-1); // free the memory allocation of the T vector
      free_dvector(P,0,ntheta-1);
      free_dvector(init_P,0,ntheta-1);
      free_dvector(val_1,0,ntheta-1); 
      free_dvector(val_2,0,ntheta-1);
      free_dvector(P_w,0,ntheta-1);
   }
}

\end{lstlisting}
\end{footnotesize}

\newpage
\section*{Appendix B: Code for Calculating Interpolated Radial Distance Values}
\label{app:appendix_B}
\addcontentsline{toc}{section}{Appendix B: Code for Calculating Interpolated Radial Distance Values}


\begin{footnotesize}
\begin{lstlisting}[language=Python]


import numpy as np
import os

theta_vals_data = []
R_vals_data = []
theta_errs_data = []
R_errs_data = []

data_values = input("Input the data file you wish to read in: ")
data_fptr = open(data_values, "r", newline=None) # opening the file with the observed data 

data_result_list = data_fptr.readlines() # creating a list containing observed data results

for result in data_result_list:
    datum = result.split( )
    for i in range(0, len(datum)):
        if i == 0:
            theta_vals_data.append(float(datum[i]))
        elif i == 1:
            R_vals_data.append(float(datum[i]))

data_fptr.close()

filenames = os.listdir(os.curdir)
filenames = os.listdir(r"/Users/matthewevans/Documents/cprogramming/Graphs/PHY3138/PHY3147/phi0_0.505_late")
file_list = []

[file_list.append(filename) for filename in filenames]


R_model_vals = [] # a list of interpolated R values

for filename in file_list:
    fpath = os.path.join(r"/Users/matthewevans/Documents/cprogramming/Graphs/PHY3138/PHY3147/phi0_0.505_late", \
    filename) # reads files from another directory
    fptr = open(fpath, "r", newline = None)
    list_of_results = fptr.readlines()
 
    data = []
    theta_vals = []
    R_vals = []
    theta_errs = []
    R_errs = [] 

    for result in list_of_results:
        data = result.split( )
        for i in range(0, len(data)):
            if i == 0:
                theta_vals.append(float(data[i]))
            elif i == 1:
                R_vals.append(float(data[i]))
        
    fptr.close()
      
    for i in range(0, len(theta_vals_data)):
        found_flag = 0
        R_model = 0
        for j in range(0, len(theta_vals)):
            if theta_vals[j] >= theta_vals_data[i] and found_flag == 0:
              

                R_model = R_vals[j-1] + ((R_vals[j] - R_vals[j-1]) / (theta_vals[j] - theta_vals[j-1])) * \
                (theta_vals_data[i] - theta_vals[j-1])
                R_model_vals.append(R_model)
               
                found_flag = 1

N = len(data_result_list)

model_data = input("Enter in the file where the RH interpolated values should be stored: ")
my_file = open(model_data, "w") # opening the interpolated R values data file e.g. 'test_model.dat'
counter = 0
for number in R_model_vals:
    if counter % N == 0 and counter != 0:
        my_file.write("\n")
    my_file.write(str(number))
    my_file.write(" ")
    
    counter += 1

my_file.close() 

\end{lstlisting}
\end{footnotesize}

\section*{Appendix C: Code for Calculating Least-Squares Differences}
\label{app:appendix_C}
\addcontentsline{toc}{section}{Appendix C: Code for Calculating Least-Squares Differences}

\begin{footnotesize}
\begin{lstlisting}[language=Python]

import numpy as np
import os
import pandas as pd

theta_vals_data = []
R_vals_data = []
theta_errs_data = []
R_errs_data = []

data_values = input("Input the data file you wish to read in: ")
fptr = open(data_values, "r", newline=None) # opening the R data values

list_of_results = fptr.readlines()

data  = []

for result in list_of_results:
    datum = result.split( )
    for i in range(0, len(datum)):
        if i == 0:
            theta_vals_data.append(float(datum[i]))
        elif i == 1:
            R_vals_data.append(float(datum[i]))
        elif i == 2:
            theta_errs_data.append(float(datum[i]))
        elif i == 3:
            R_errs_data.append(float(datum[i]))


fptr.close()

R_vals_model = []

model_data = input("Enter in the file with the RH model data: ")
fptr2 = open(model_data, "r", newline=None) # opening the interpolated R values 'test_model.dat'

list_of_results2 = fptr2.readlines()

data  = []

for result in list_of_results2:
    datum = result.split( )
    for i in range(0, len(datum)):
        R_vals_model.append(float(datum[i]))

        
fptr2.close()


data_values = np.array(R_vals_data) # creating an array for the R data values
N = len(data_values)
print(N)

data_errors = np.array(R_errs_data) # creating an array for the R errors

model = np.array(R_vals_model) # creating a 2D array for the model (interpolated) values
shape = (101, N)
model_values = model.reshape(shape) 


residual_values = []
ls_values = []
for i in range(0, len(model_values)):

    residual = 0
    ls_value = 0
    for j in range(0, N):
        residual = data_values[j] - model_values[i][j]
        ls_value += (residual ** 2) # remember to divide by errors when we get data!
        if j == (N - 1):
            residual_values.append(residual)
            ls_values.append(ls_value)


min_value = min(ls_values)

lowest_res_vals = []
for i in range(0, len(ls_values)): # prints out the minimum values and index number
    if ls_values[i] == min_value:
        print("The minimum value is: {:f} " .format(ls_values[i]))
        print("The corresponding (index number) RH is: {:d} " .format(i))

        lowest_res = 0
        for j in range(0, N):
            lowest_res = data_values[j] - model_values[i][j]
            lowest_res_vals.append(lowest_res)

output_file = input("Enter in the file name you wish to write ls values to: ") # e.g. 'ls_test.dat'
my_file = open(output_file, "w")

for number in ls_values:
    my_file.write(str(number))
    my_file.write("\n")

my_file.close()


\end{lstlisting}
\end{footnotesize}


\end{document}