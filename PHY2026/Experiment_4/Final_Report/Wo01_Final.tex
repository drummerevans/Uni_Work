% Comments start with % (percent) character and last till the end of the line.
%
% The line below tells TeXworks editor to use pdflatex for compilation
% of this document; remove it if you want to use another engine 
%
%!TEX program = pdflatex
%
% LaTeX2e document starts with \documentclass[options]{<class-name>}
% <class-name> can be one of the standard LaTeX document classes: 
% article, report or book, or some other specialised class.
%
\documentclass{article}
%
% Preamble of LaTeX document is everything before \begin{document}.
% Preamble is used to load extension packages and to set up global 
% parameters and configuration for the entire document.
%
% Extension packages providing additional functionality
\usepackage{amsmath}       % additional math environments
\usepackage{graphicx}      % graphics import from external files 
\usepackage{epstopdf}      % automates .eps to .pdf conversion 
% epstopdf package may require --shell-escape option to pdflatex
\usepackage{booktabs}      % table typesetting additions
\usepackage{siunitx}       % number and units formatting
\usepackage{caption}       % customisation of captions
\usepackage{url}           % format url addresses
\usepackage{abstract}		% allows formatting of abstract
\usepackage[margin=0.5in]{geometry}
%\usepackage{tikz,pgfplots} % diagrams and data plots
%
% set up caption options
\captionsetup{margin=12pt,font=small,labelfont=bf}
%
%removes abstract title
\renewcommand{\abstractname}{}
%sets abstract margins
\setlength{\absleftindent}{10mm}
\setlength{\absrightindent}{10mm}
%
% global options for siunitx
%\sisetup{seperr,repeatunits=false,per=symbol}
%
% some handy commands for referencing;
% the optional argument overrides the default label, e.g.
% \figref[FIG.~]{fig:label}
\newcommand{\figref}[2][\figurename~]{#1\ref{#2}}
\newcommand{\tabref}[2][\tablename~]{#1\ref{#2}}
\newcommand{\secref}[2][Section~]{#1\ref{#2}}


% The document content starts with \begin{document} 
% and is finished with \end{document}
%
\begin{document}
\title{Mechanical and Electromagnetic Waves} % fill in the title here
\author{Matthew Evans}% fill in your name here
\date{14\textsuperscript{th} February 2019} % date of the report
%\twocolumn[	% makes title and abstract appear over entire page width
\maketitle % formats the title
%\begin{onecolabstract}
%\noindent

%Enter abstract here.


\section{Introduction}
\label{sec:introduction}

The various properties and characteristics of waves were investigated in these experiments. These physical phenomena are at the heart of many transmission processes; such as the propagation of sound through a medium responsible for audio communication and the transmission of electromagnetic waves in a vaccuum which enables visible light from the Sun to travel to Earth. In addition, quantum mechanics has made the connection between the wave-like nature exhibited by particles and the particle-like nature of waves. Interesting developments have been made due to the study of waves and this is at the forefront of modern technologies.

\vspace{2mm}
\noindent
A wave can be thought of as a time-varying disturbance that propagates through a medium or a vacuum. This can be described mathematically as a \textit{wave function} $\psi(x, t)$

\begin{equation}
\label{eq:wave}
\psi(x, t) = \psi_0 cos(kx - \omega t)
\end{equation}

\vspace{2mm}
\noindent
where, $\psi_0$ is the wave amplitude, $x$ is the spatial position of the propagating wave at time $t$, $k$ is the wave vector related to the wavelength, $\lambda$ by $k = \frac{2\pi}{\lambda}$, and $\omega$ is the angular frequency related to the time period, $T$, of the wave by $\omega = \frac{2\pi}{T}$. Equation \eqref{eq:wave} only considers the real part of the propagating wave. In most instances, waves also have an imaginary part. From Euler's identity equation \eqref{eq:wave} can be re-expressed as a complex exponential

\begin{equation}
\label{eq:complex-wave}
\psi(x, t) = \psi_0 e^{i(kx - \omega t)}
\end{equation}

\vspace{2mm}
\noindent
and all the symbols have their usual meaning as defined for equation \eqref{eq:wave}.

\vspace{2mm}
\noindent
The various characteristics of waves is the same for different types of waves \cite{Paper01}. Making use of this physical concept, the first of these experiments investigates wave phenomena using mechanical water waves in a ripple tank. In particular reflection, refraction, diffraction and interference were studied. The latter two phenomena were investigated further in the second experiment where a laser light was diffracted through single and double slits to investigate interference and diffraction patterns. Furthermore, extensions to these experiments looked into other wave phenomena such as the Doppler effect, multiple slit diffraction and interference and how water wave speed depends on the depth.


\section{Theory}
\label{sec:theory}

The speed of mechanical waves often depend upon the medium which they travel, in addition to many other factors. For water waves, the speed depends on depth. For \textit{shallow depths}, this is given by \cite{Web01}

\begin{equation}
\label{eq:water-waves}
v \approx \sqrt{gd}
\end{equation}

\vspace{2mm}
\noindent
where $g$ is the acceleration due to gravity and $d$ is the measured depth of the water used in the ripple tank. For the interested reader, a more detailed analysis of water waves can be found at which analysis \cite{Web01}. 

\vspace{2mm}
\noindent
When a wave front approaches at an angle to a boundary \textit{reflection} occurs. This is when the angle of incidence, $\theta_i$ is equal to the angle of reflection, $\theta_r$ at the boundary

\begin{equation}
\label{eq:reflection}
\theta_i = \theta_r
\end{equation}

\vspace{2mm}
\noindent
\figref{fig:reflection_refraction} describes this pictorially.

\vspace{2mm}
\noindent
If the wave passes between two different media and changes speed, as a consequence, it will therefore change direction. This is known as \textit{refraction}. \figref{fig:reflection_refraction} shows a physical representation for this phenomena.

\begin{figure}[h]
\centering
\includegraphics[scale=0.4]{"reflection_refraction".pdf}
\caption{A diagram demonstrating the physical processes of reflection and refraction as an incident wave front encounters a boundary. The direction of travel for the incident and reflected waves are shown by the black arrows, whilst the direction of travel for the refracted wave is shown by the red arrow. The two media have refractive indices $n_i$ and $n_t$. The angles, $\theta_i$, $\theta_r$ and $\theta_t$ correspond to the angles of incidence, reflection and refraction respectively between direction of travel and the normal to the boundary.}
\label{fig:reflection_refraction}
\end{figure}

\vspace{2mm}
\noindent
Refraction of a wave between two different media is described by Snell's Law

\begin{equation}
\label{eq:refraction}
n_i sin{\theta_i} = n_t sin{\theta_t}
\end{equation}


\vspace{2mm}
\noindent
where $n_i$ and $n_t$ are the refactive indices of the inital and final media respectively, $\theta_i$ is the angle of incidence and $\theta_t$ is the angle of refraction between two different media. This is usually used when considering the refraction of electromagnetic waves at a boundary. However, the physics of waves is the same for different types of waves and, more recently, it has been found \cite{Web02} that water waves, when refracted obey Snell's Law \eqref{eq:refraction}.


\vspace{2mm}
\noindent
\textit{Diffraction} occurs when a wave `bends' as it passes through a gap (or slit) between two obstacles or when the wave travels around a barrier. An illustration of the diffraction of light is given in \figref{fig:diffraction} when incident light waves encounter a single slit.

\begin{figure}[h]
\centering
\includegraphics[scale=0.365]{"diffraction".pdf}
\caption{An illustration demonstrating the diffraction of light from a single slit and defining some terms. The incident light wave approaches a slit of width $a$ and two beams are considered from the Huygen's wavelets \cite{Book01} formed as a result of passing through the slit. These two beams arrive at a point $P$ creating an interference pattern on the screen and the bottom ray is diffracted through by an angle $\theta$. The distance between the plane of the slit and the screen is $L$ and the vertical distance between the origin, $O$ and point $P$ is $x$. }
\label{fig:diffraction}
\end{figure}

\vspace{2mm}
\noindent
The intensity of diffracted light from a single slit into an angle $\theta$ \cite{Paper02} is given by

\begin{equation}
\label{eq:intensity1}
I = I_0 \frac{sin^2{\alpha}}{\alpha^2}
\end{equation}

\vspace{2mm}
\noindent
where $\alpha = \frac{\pi asin{\theta}}{\lambda}$, $\lambda$ is the wavelength of the light and $a$ is the slit width.

\vspace{2mm}
\noindent
For two slits, the intensity of diffracted light into an angle $\theta$ \cite{Paper02} is

\begin{equation}
\label{eq:intensity2}
I = I_0 \frac{sin^2{\alpha}}{\alpha^2} cos^2{\delta}
\end{equation}

\vspace{2mm}
\noindent
where $\delta = \frac{\pi dsin{\theta}}{\lambda}$ and $d$ is the separation distance between the two slits. A derivation for equations \eqref{eq:intensity1} and \eqref{eq:intensity2} involving a geometrical approach and phasor diagrams can be found by \cite{Book01}. These equations can also be dervied by considering Fourier transforms and the convolution theorem this approach can be found from \cite{Web03}.

\vspace{2mm}
\noindent
Using equation \eqref{eq:intensity1}, the intensity minima for a single slit occurs when $\alpha$ is a multiple of $\pi$. This therefore means that

\begin{equation}
\label{eq:min_intensity}
sin{\theta} = \frac{m \lambda}{a} \quad,\quad(m = \pm1, \pm2, ...)
\end{equation}

\vspace{2mm}
\noindent
If L'H{\^o}pital's rule \cite{Book01} is applied to equation \eqref{eq:intensity1} in the limit as the intensity at $\alpha \rightarrow 0$ is found to be $I = I_0$ as expected. When the detector to slit distance is big, the small angle approximation can be used. Equation \eqref{eq:min_intensity} can then be approximated to

\begin{equation}
\label{eq:small_angle}
\theta \approx \frac{m\lambda}{a} \quad,\quad(m = \pm1, \pm2, ...)
\end{equation}


\vspace{2mm}
\noindent
The intensity maxima of a single slit diffraction pattern can be approximately found by using equation \eqref{eq:intensity1} and realising that they occur when the sine function is a maximum ($\pm 1$) \cite{Book01}. In other words, when

\begin{equation}
\label{eq:alpha_max}
\alpha = \pm\bigg(m + \frac{1}{2}\bigg)\pi \quad,\quad(m = 0, 1, 2, ...)
\end{equation}

\vspace{2mm}
\noindent
However, upon further analysis of applying differentiation to equation \eqref{eq:intensity1} and setting equal to zero to find the maxima it is found that the there is no maxima at $m = 0$ \cite{Book01}. Therefore, when equation \eqref{eq:alpha_max}, with $m \neq 0$, is subsituted into equation \eqref{eq:intensity1} the intensity maxima, $I_m$ is approximately given by \cite{Book01}

\vspace{2mm}
\noindent
\begin{equation}
\label{eq:max_intensity}
I_m \approx \frac{I_0}{\big(m + \frac{1}{2}\big)^2\pi^2} \quad,\quad(m =  \pm1, \pm2, ...)
\end{equation}


\section{Method}
\label{sec:method}

\subsection{Waves in a Ripple Tank}
\label{ssec:ripple-method}

\vspace{2mm}
\noindent
The ripple tank was placed on an up-turned crate and the camera was carefully mounted on the tripod directly above the ripple tank. These factors were needed to ensure that the camera set-up was in the best possible position for obtaining the required images of the water waves. Then the height of the tripod was noted along with the camera `zoom' focus in order to keep consistency when taking photographs of the waves generated. The tray was then partially filled with water up to the sloping part of the foam \cite{Paper01} to reduce reflections from the edges. The depth of the water was then measured using a ruler and the uncertainty noted. Then a stroboscopic LED was used illuminate this from underneath the tray. The stroboscopic LED `froze' the movement of the waves so that these can be captured by the camera and the resulting images can then be analysed. \figref{fig:ripple_setup} shows how the images of the travelling wave fronts were generated from the ripple tank.

\begin{figure}[h]
\centering
\includegraphics[scale=0.42]{"ripple_physics".pdf}
\caption{A side-view of the ripple tank. This shows how the strobing light source in conjunction with the generated waves produced images on the screen, placed directly above the ripple tank. A camera could then be set-up directly above the screen to capture images of the resulting wave fronts.}
\label{fig:ripple_setup}
\end{figure}

\newpage
\vspace{2mm}
\noindent
The frequency of the vibrator was first set to 20 Hz. The height of the dipper was carefully adjusted so that the best quality images were obtained from the camera. These images were then converted to distances in cm using an image processing program. A calibration image was taken of a rule in by the camera to get a conversion between image distance and `real' distance. The uncertainties in the image processing program in the distances were noted and these were also converted to real distances. \figref{fig:ruler} shows the calibration image used. 

\begin{figure}[h]
\centering
\includegraphics[scale=0.42]{"ruler".pdf}
\caption{A photograph of a ruler was used to convert the image distances to real distances from the camera. The points between a 1cm mark on the ruler were measured in the image processing program and the distance noted. Then this was used as a conversion factor to find the wavelengths of subsequent images taken of the wave fronts in the experiment. $x$ and $y$ are the image distances and these were used to determine the length of the red line which corresponds to 1cm in real distance.}
\label{fig:ruler}
\end{figure}

\vspace{2mm}
\noindent
By using the $x$ and $y$ image program distances the length of the red line that corresponds to 1cm in real distance in \figref{fig:ruler}, was determined using 

\vspace{2mm}
\noindent
\begin{equation}
\label{eq:pythagoras}
d' = \sqrt{x^2 + y^2}
\end{equation}

\vspace{2mm}
\noindent
where $d'$ is the image program distance. The conversion factor was then found to be $0.66 \pm 0.1$cm corresponded to 1cm in real distance. Therefore, all image distances were divided by 0.66 i.e. 

\vspace{2mm}
\noindent
\begin{equation}
\label{eq:conversion}
d = \frac{d'}{0.66}
\end{equation}

\vspace{2mm}
\noindent
to gain the real image distance, $d$, in centimetres.

\vspace{2mm}
\noindent
Various different objects and obstacles were placed in the water enabling various wave phenomena to be investigated:

\begin{enumerate}
  \item \textbf{The speed of water waves:} Plane waves generated from the vibrator were investigated to determine the wavelength of the waves at different frequencies. Three images were obtained for each frequency to gain more reliable results. The wavelengths were then found by using the image processing program in conjunction with equation \eqref{eq:pythagoras}. Then, an average of the three wavelengths for each frequency was obtained, along with propagation of the associated uncertainties, to gain a more reliable value for the wavelength. These wavelengths were then converted to real distances using equation \eqref{eq:conversion} and the uncertainties were also obtained. The speed of the water waves was then determined by plotting a graph of wavelength, $\lambda$ against inverse frequency (time), $1/f$. This was then compared to the theoretical value of water waves in \textit{shallow depths} calculated by using equation \eqref{eq:water-waves}. 
  \item \textbf{Reflection:} A barrier was placed in the tray at an angle to the incident wave fronts. The angle of incidence, $\theta_i$ and angle of reflection, $\theta_r$ was then measured (using an image processing program) to verify relation \eqref{eq:reflection} for 20 Hz waves. In addition, lower frequency wave fronts were also investigated qualitatively to see how reflection changed at different frequencies.
  \item \textbf{Refraction:} The velocity of water waves depend on the depth. Therefore, an object was be placed underwater and this resulted to change the of velocity of the water waves as they pass over it. This caused a change in direction of the waves and hence refraction could be investigated.
  \item \textbf{Single-slit interference:} A barrier with a gap was placed in front of the dipper \cite{Paper01}. The gap width was then changed and the camera was used to take pictures at these different widths. In addition, the frequency was also adjusted while keeping the gap width fixed to investigate this further. Images of the maxima and minima from the resulting interference patterns were then studied.
  \item \textbf{Double-slit interference:} Two gaps were created from barriers in the tray and the Young's fringes generated from the interference patterns were investigated. 
  \item \textbf{Double-source Interference:} Two choerent sources will be observed using the double dipper or mechanical vibrators \cite{Paper01}. With the double dipper, Young's diffraction pattern will be observed and with the mechanical vibrator standing waves will be generated for analysis.
  \item \textbf{Extensions:} Various other wave properties will potentially be investigated:
  	\begin{itemize}
     \item Doppler Effect by moving an external vibrator as it is generating waves.
     \item Speed of water waves as a function of depth.
     \item Diffraction due to waves passing \textit{around} a barrier.
   \end{itemize}
\end{enumerate}

\subsection{Diffraction and Interference}
\label{ssec:diffraction-method}

For investigating the diffraction and interference of electromagnetic waves, a 650nm laser light was used \cite{Paper02} and a rotating slit wheel with various slits was used. Also a light sensor with a variable aperture connected to a computer via an interface box allowed position and intensity readings to be stored. All the settings were then set-up on the computer software in order to obtain the relevant experimental data. 

\vspace{2mm}
\noindent
Before any measurements were taken two preliminary measurements were taken. First, the background intensity was recorded from the sensor and was then used to correct the subsequent intensity readings. Then the laser light was used to illuminate the sensor with the aperture open at a maximum (and with no slit wheel in place) so that it was receiving close to $100\%$ intensity \cite{Paper01}. This was then later used to normalise all the intensity readings.

\vspace{2mm}
\noindent
After the preliminary measurements were taken, the experiment was first set-up as shown in \secref{sec:theory} \figref{fig:diffraction}. The slit wheel placed at a length if $L = 0.5$m from the screen to investigate diffraction patterns for a single slit. This length was then kept fixed throughout all the subsequent experiments. A light sensor was mounted on a lateral slider was connected to a computer via an interface box. This sensor was placed at the screen distance from the single slit as shown in \figref{fig:diffraction}. The lateral slider could then be used to carefully slide the sensor across the length of the screen, in the `$x$' direction indicated in \figref{fig:diffraction}. The interface box then recorded the sensor's position and received intensity and stored this on the computer for further analysis. This experiment was repeated for different slit widths and sensor aperture sizes to investigate the intensity at various positions in more detail.

\newpage
\vspace{2mm}
\noindent
Once the appropriate investigations for the single slit at various widths and aperture sizes had been investigated, double slit interference patterns were investigated. The experiment was adjusted so, the set-up was as shown in \figref{fig:double_diffraction}.

\begin{figure}[h]
\centering
\includegraphics[scale=0.4]{"double_diffraction".pdf}
\caption{An illustration demonstrating the diffraction of light from double slits and defining some terms. The incident light wave approaches two slits of width $a$ separated by a distance $d$. The two beams are considered from the Huygen's wavelets \cite{Book01} formed as a result of passing through the slits. These two beams from the slits are assumed to be parallel as $L >> x$ and arrive at a point $P$ creating an interference pattern on the screen. The rays are diffracted through by an angle $\theta$. The distance between the plane of the slit and the screen is $L$ and the vertical distance between the origin, $O$ and point $P$ is $x$.}
\label{fig:double_diffraction}
\end{figure}

\vspace{2mm}
\noindent
As for the single slit experiment, a sensor on a lateral slider was used to measure the position and corresponding intensities of the diffraction patterns. Different slit widths and separations alongs with different sensor aperture sizes were used to investigate the resulting interference patters generated from the diffracted laser light.

\vspace{2mm}
\noindent
Further diffraction experiments will then follow: single-slit, double-slit and multiple-slit patterns will be studied. In addition, different shape apertures will also be used to study generated diffraction patterns.   


%\section{Results}
%\label{sec:results}




%\section{Discussion}
%\label{sec:discussion}




%\section{Conclusion}
%\label{sec:conclusion}


\begin{thebibliography}{9}
\bibitem{Paper01} College of Engineering Mathematics and Physical Sciences, University of Exeter, PHY2026, \textit{Waves in a Ripple Tank Worksheet} (Accessed 8\textsuperscript{th} February 2019).
\bibitem{Paper02} College of Engineering, Mathematics and Physical Sciences, University of Exeter, PHY2026, \textit{Diffraction and Interference Worksheet} (Accessed 8\textsuperscript{th} February 2019).
\bibitem{Book01} Young, Hugh D and Freedman, Roger A, \textit{University Physics}, 13\textsuperscript{th} Edition, Chapter 36, pages 1312 - 1322, 2014.
% \bibitem{Book02} Young, Hugh D and Freedman, Roger A, \textit{University Physics}, 13\textsuperscript{th} Edition, Chapter 20, pages 728 - 740, 2014.
\bibitem{Web01} \url{http://www.dartmouth.edu/~cushman/books/EFM/chap4.pdf} (Accessed 23\textsuperscript{rd} February 2019.)
\bibitem{Web02} \url{https://physicsworld.com/a/working-with-water-waves/} (Accessed 9\textsuperscript{th} February 2019.)
\bibitem{Web03} \url{http://people.ucalgary.ca/~lvov/471/labs/fraunhofer.pdf} (Accessed 9\textsuperscript{th} February 2019.)

\end{thebibliography}

\end{document}