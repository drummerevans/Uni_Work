% Comments start with % (percent) character and last till the end of the line.
%
% The line below tells TeXworks editor to use pdflatex for compilation
% of this document; remove it if you want to use another engine 
%
%!TEX program = pdflatex
%
% LaTeX2e document starts with \documentclass[options]{<class-name>}
% <class-name> can be one of the standard LaTeX document classes: 
% article, report or book, or some other specialised class.
%
\documentclass{article}
%
% Preamble of LaTeX document is everything before \begin{document}.
% Preamble is used to load extension packages and to set up global 
% parameters and configuration for the entire document.
%
% Extension packages providing additional functionality
\usepackage{amsmath}       % additional math environments
\usepackage{graphicx}      % graphics import from external files 
\usepackage{epstopdf}      % automates .eps to .pdf conversion 
% epstopdf package may require --shell-escape option to pdflatex
\usepackage{booktabs}      % table typesetting additions
\usepackage{siunitx}       % number and units formatting
\usepackage{caption}       % customisation of captions
\usepackage{url}           % format url addresses
\usepackage{abstract}		% allows formatting of abstract
\usepackage[margin=0.7in]{geometry}
%\usepackage{tikz,pgfplots} % diagrams and data plots
%
% set up caption options
\captionsetup{margin=12pt,font=small,labelfont=bf}
%
%removes abstract title
\renewcommand{\abstractname}{}
%sets abstract margins
\setlength{\absleftindent}{10mm}
\setlength{\absrightindent}{10mm}
%
% global options for siunitx
%\sisetup{seperr,repeatunits=false,per=symbol}
%
% some handy commands for referencing;
% the optional argument overrides the default label, e.g.
% \figref[FIG.~]{fig:label}
\newcommand{\figref}[2][\figurename~]{#1\ref{#2}}
\newcommand{\tabref}[2][\tablename~]{#1\ref{#2}}
\newcommand{\secref}[2][Section~]{#1\ref{#2}}


% The document content starts with \begin{document} 
% and is finished with \end{document}
%
\begin{document}
\title{At01 Atomic Spectroscopy \& Electron Spin Resonance} % fill in the title here
\author{Matthew Evans}% fill in your name here
\date{15\textsuperscript{th} November 2018} % date of the report
%\twocolumn[	% makes title and abstract appear over entire page width
\maketitle % formats the title

\begin{onecolabstract}
\noindent
These experiments investigated the energy transitions involved with electron transitions within the hydrogen atom and energy differences associated with electron spin state changes. The value of the Rydberg constant determined from the atomic spectroscopy experiment was $13.1 \pm 0.2$ eV and the one given from theory is $13.6$ eV \cite{Paper01}. The spectroscopic splitting factor obtained from investigating electron spin resonance was $1.9733 \pm 0.0119$ and the one given from theory is $2.0023$ \cite{Paper02}. These both the experimental values were fairly close to the ones given from theory; however, both of the theoretical values were outside the uncertainty range of the experimental ones. By studying the values of these key atomic constants, many interesting and fundamental concepts, such analysing emission spectra from stars and quantum mechanical properties of atoms, can be investigated in much more detail.

\end{onecolabstract}
\vspace{5mm} % takes one free line before rest of text


\section{Introduction}
\label{sec:introduction}

Two key atomic concepts were be investigated during this experiment: the atomic spectrum of hydrogen and the spin resonance of electrons. These two experiments are closely linked and are of key importance when studying atomic characteristics. By studying the electron spin we can find that some energy levels of atoms obtained from spectroscopic techniques are further split into smaller energy levels called \textit{multiplets} \cite{Book01}. From this other phenomena can be more accurately investigated, such as investigating the chemical composition of stars. 

\vspace{2mm}
\noindent
Electrons occupy energy levels in atoms that orbit around the nucleus. When electrons gain energy, for example from a collision with an external electron or absorption of a quantum  `packet' of energy called a \textit{photon}, they move up to the next energy level and are then said to be \textit{excited}. After a  period of time, they de-excite back to their original energy level emitting a photon equivalent to the energy difference of the two levels 

\begin{equation}
\label{eq:energy-diff}
h \nu = E_f - E_i
\end{equation}

\vspace{2mm}
\noindent
where $h = 6.63 \times 10^{-34}$ Js, is the Planck constant, $\nu$ is the photon frequency, $E_f$ is the excited energy level, $E_i$ is the original equilibrium energy level. These photon energies can act as `finger prints' for atoms and hence the particular type of atomic species can be identified by analysing the emitted or absorption spectrum of an atom. Where the absorption spectrum will have particular wavelengths of light missing from a continuous spectrum due to the absorption of photons by the atom from the excitation process. 
In this experiment we will be analysing the \textit{emitted} spectrum of hydrogen \cite{Paper01}.

\vspace{2mm}
\noindent
In addition, all electrons have a spin angular momentum associated with them \cite{Paper02}. The electron spin component is given by

\begin{equation}
\label{eq:spin}
S_z = m_s\hbar
\end{equation}

\vspace{2mm}
\noindent
where $S_z$ is the electron spin angular momentum by considering the z-component, $m_s = \pm\frac{1}{2}$ refers to the spin quantum number either `spin up' (positive) or `spin down' (negative) respectively and $\hbar = \frac{h}{2\pi}$. This concept can be likened to an anaology of a spinning sphere \cite{Book01} where the spin angular momentum is the momentum associated with the body about z-axis. The magnitude of the spin angular momentum is

\begin{equation}
\label{eq:spin-magnitude}
S = \hbar\sqrt{s(s+1)}
\end{equation}

\vspace{2mm}
\noindent
where $s = \mid{m_s}\mid = \frac{1}{2}$ and is the spin quantum number. Because electrons have a charge associated with them, they will be affected as they pass through a magnetic field. %Therefore, the spinning motion of the electron will be changing as it passes through the field.
 This experiment investigates this effect further and obtains some fundamental information from this process.

\section{Theory}
\label{sec:theory}

The energy levels of a simplified diagram for the hydrogen atom is shown in \figref{fig:levels}.

\begin{figure}[h]
\centering
\includegraphics[scale=0.48]{"levels".pdf}
\caption{A simplified diagram showing the energy levels of the hydrogen atom. Where n is the principle quantum number. n = 1 is the ground state and n = $\infty$ correspons to the ionisation energy required to liberate the bound electron \cite{Paper01}.}
\label{fig:levels}
\end{figure}

\vspace{2mm}
\noindent
These energy levels can be determined by using the Bohr model or by solution to the Schr\"{o}dinger equation \cite{Paper01} and is given by

\begin{equation}
\label{eq:energy-levels}
E_n = -\frac{Ry}{n^2}
\end{equation}

\vspace{2mm}
\noindent
where $Ry$ is the Rydberg constant and has a value of 13.6 eV and $n$ is the energy level number in the atom. The experiment considered transitions to the $n = 2$ energy level, i.e. the Balmer Series, for the photon energies as the first few photons in the Balmer Series lie in the visible region of light. This because the first first few energy transitions for this series is in the visible region and can easily be observed. The transitions occur by exposing the the spectral tube to an electric field and this ionises the gas releasing electrons \cite{Paper01}. These free electrons then collide with other electrons bound by hydrogen atoms, as mentioned in \secref{sec:introduction}. This excites the bound electrons in the atoms to an excited state. Then, these electrons would de-excite via the emission of a photon, producing a constant output beam. Therefore by considering the relation \eqref{eq:energy-diff}, equation \eqref{eq:energy-levels} becomes

\begin{equation}
\label{eq:balmer-series}
E_n = -Ry\Bigg(\frac{1}{n^2} - \frac{1}{2^2}\Bigg) = Ry\Bigg(\frac{1}{2^2} - \frac{1}{n^2}\Bigg)
\end{equation}

\vspace{2mm}
\noindent
For a more detailed derivation of the energies levels for the hydrogen atom using the Bohr model please refer to \cite{Book02}.

\vspace{2mm}
\noindent
Electrons are negatively charged particles, therefore they have an associated magnetic moment

\begin{equation}
\label{eq:magnetic-moment}
\boldsymbol{\mu} = -g\frac{e}{2m}\textbf{S}
\end{equation}

\vspace{2mm}
\noindent
\textbf{$\mu$} is the magnetic moment, $g$ is the spectroscopic splitting factor, $e$ and $m$ are the charge and mass of the electron respectively \cite{Paper02}. As mentioned in \secref{sec:introduction} the spinning motion of the electron is effected as it passes through a magnetic field \textbf{B} because the magnetic dipoles of the electron and field interact with each other. This means that the electron will precess about the magnetic field, \textbf{B}, in one of two orientations as shown in \figref{fig:spin} \cite{Paper02}.

\begin{figure}[h]
\centering
\includegraphics[scale=0.85]{"spin".pdf}
\caption{Left: orientation of electron `spin up' and `spin down' are given in a magnetic field \textbf{B}. Right: dependence of the two energy states on \textbf{B} is shown \cite{Paper02}.}
\label{fig:spin}
\end{figure}

\newpage
\vspace{2mm}
\noindent
The magnetic energy is given by \cite{Paper02}

\begin{equation}
\label{eq:potential1}
U = -\boldsymbol{\mu} \cdot \textbf{B} = - \mu_zB
\end{equation}

\vspace{2mm}
\noindent
Where $\mu_z$ is the z-component, $\mu_z = -g\frac{e}{2m}S_z$, of the magnetic moment. By substituting \eqref{eq:spin} into \eqref{eq:potential1}, we now have

%\begin{split} for splitting equations
% & \\ = 
% & =
%\end{split}

\begin{equation}
\label{eq:potential2}
U = g\frac{e}{2m}(m_s\hbar)B = g\mu_BB
\end{equation}

\vspace{2mm}
\noindent
where $\mu_B = \frac{e\hbar}{2m}$ is the Bohr magneton. Therefore, from \figref{fig:spin} and \eqref{eq:potential2} we can see that the difference in energy be determined by \cite{Paper02}

\begin{equation}
\label{eq:potential-difference}
\Delta U = g\mu_BB\Big(\frac{1}{2} - \Big(-\frac{1}{2}\Big)\Big) = g\mu_BB.
\end{equation}

\vspace{2mm}
\noindent
For further information regarding this derivation arriving it equation \eqref{eq:potential-difference}, please refer to \cite{Book01}. By considering the electron energy changes involved with the atomic spectroscopy mentioned earlier, we can therefore deduce that the electron can either go up or down an energy state by photon absorption or emission receptively. Hence the energy of the photon required for such a transition is given by

\begin{equation}
\label{eq:frequency-equation}
h\nu = g\mu_BB
\end{equation}

\vspace{2mm}
\noindent
where $\nu$ is again the frequency of the absorbed or emitted photon. 

\newpage
\section{Method}
\label{sec:method}

\subsection{Atomic Spectroscopy}
\label{ssec:atomic-method}
The exerimental set-up is shown in \figref{fig:atom_picture}. 

\begin{figure}[h]
\centering
\includegraphics[scale=0.48]{"atom_setup".pdf}
\caption{Experimental set-up for the atomic spectroscopy experiment.}
\label{fig:atom_picture}
\end{figure}

\noindent
First the Hydrogen spectral tube, marker and ruler used to determine $l$ were carefully set-up. The tube was positioned in the centre of the ruler and the measurement taken along with the uncertainty. Then, the diffraction grating was positioned to make the distance from the spectral tube, $d$, as big as possible. This was carefully measured, along with the associated uncertainty. The low pressure hydrogen gas in the tube was then exposed to an electric field by carefully adjusting the voltage source as shown in \figref{fig:atom_picture} so that the spectral tube produced as constant output beam ready for observation. These emission lines were then observed through a diffraction grating as shown in \figref{fig:grating_plan} and the marker shown in \figref{fig:atom_picture} was adjusted so that the spectral line position could be measured. 

\begin{figure}[h]
\centering
\includegraphics[scale=0.35]{"grating_plan".pdf}
\caption{A plan view of the experimental set-up \cite{Paper01}.}
\label{fig:grating_plan}
\end{figure}

\vspace{2mm}
\noindent
The difference between the central position of the tube and measured position was then determined along with the associated uncertainties in order to get a value for $l$. The diffraction grating was then moved in 50mm decrements towards the tube until the lines spectral lines were unclear to view. After this had been undertaken, the diffraction grating was moved back to its original value of $d$ and the above process was then undertaken a total three times to improve accuracy and reduce errors.

\vspace{2mm}
\noindent
The next step was then to use the marker, as shown in \figref{fig:atom_picture}, to measure the length $l$ of mercury emission lines along the ruler using the mercury lamp in place of the spectral tube in \figref{fig:grating_plan} to determine the grating spacing constant $d_g$. This process was similar to measuring the hydrogen emission lines as outlined above. These measurements were performed for the three furthest distances of $d$ as these were the most accurate. This was repeated a total of three times to reduce uncertainty and improve accuracy. An average of these obtained values of $d_g$ was taken along with its unceratinty carefully determined. Then, by using the known wavelengths of mercury given in \cite{Paper01} and the diffraction grating equation 

\begin{equation}
\label{eq:grating-equation}
m\lambda = d_gsin\alpha \implies \lambda = d_g\frac{l}{\sqrt{l^2+d^2}}
\end{equation}

\vspace{2mm}
\noindent
we can rearrange to determine $d_g$ the grating constant and since we were observing the first bright fringe order, $m$ = 1 in this case. The obtained value of $d_g$, $1.69 \pm 0.02 \times 10^{-6}$m was then compared against the known value $1.67 \times 10 ^{-6}$m \cite{Paper01} and this shows that this is correct with the known value lying within its associated error. Therefore, this check verified that the obtained value of $d_g$ could be used for further calculations. Next, by using the obtained values of $l$ and $d$ an appropriate graph was plotted of $l$ vs $\sqrt{d^2 + l^2}$. The gradient of this graph, $\frac{\lambda}{d_g}$ and the gradient error was used in conjunction with the determined value of $d_g$ and its associated error to determine the wavelength, $\lambda$ of the sample and error respectively for each of the bright fringes observed.

\vspace{2mm}
\noindent
Once the wavelengths were determined, the energies can were obtained from the relation 

\begin{equation}
\label{eq:energy-wavelength}
E = \frac{hc}{\lambda}
\end{equation}

\vspace{2mm}
\noindent
where, $c$ is the speed of light and $\lambda$ is the wavelength of the emission lines. Then a graph can be plot using \eqref{eq:balmer-series} to determine a value for the Rydberg constant $Ry$.

\subsection{Electron Spin Resonance}
\label{ssec:electron-method}

The sample tested was an organic compound called DPPH (1, 1-diphenyl-2-picrylhydrazil) molecule \cite{Paper02}. The experimental geometry is shown in \figref{fig:electron_setup}. 

\begin{figure}[h]
\centering
\includegraphics[scale=0.4]{"electron_setup".pdf}
\caption{The experimental geometry for investigating electron spin resonance of DPPH \cite{Paper02}.}
\label{fig:electron_setup}
\end{figure}

\vspace{2mm}
\noindent
A series of preliminary experiments were conducted before the main experiment. First, the magnetic field produced by the Helmholtz pair of coils were be measured over a range of DC currents using a Gauss meter. A relation describing the magnetic field for this set-up in \figref{fig:electron_setup} is given by

\begin{equation}
\label{eq:magnetic-equation}
B = \mu_0\Bigg(\frac{4}{5}\Bigg)^{3/2}\frac{nI}{r}
\end{equation}

\vspace{2mm}
\noindent
where $\mu_0$ is the permeability of free space, $n$ is the number of turns per coil ( $n$ = 320 in this case), $I$ is the current through each coil and $r$ is the distance between the two coils in this case it is the coil radius (6.75cm) \cite{Paper02}. However, the actual distance was greater than this and $r = 7.5 \pm 0.7$cm in this case, due to the width of the radio frequency oscillator having to fit between the two coils to acquire the desired experimental set-up as shown in \figref{fig:electron_picture}. More will be mentioned about this discrepancy in \secref{sec:discussion}. Once a range of magnetic field measurements had been obtained in conjunction with the corresponding currents, these currents were used in \eqref{eq:magnetic-equation} to obtain the magnetic field. When comparing the values of the magnetic field measured using the Gauss meter to the ones obtained by relation \eqref{eq:magnetic-equation}, they were of the same order of magnitude. Therefore, this verified that \eqref{eq:magnetic-equation} is a valid way of determining the magnetic field from a range of currents and the next steps in the experimental method could proceed.

\vspace{2mm}
\noindent
The final preliminary check was to see that removing energy from a circuit will put load on the circuit and reduce the output voltage \cite{Paper02}.

\begin{figure}[h]
\centering
\includegraphics[scale=0.43]{"pre_exp2".pdf}
\caption{Preliminary Experimental set-up demonstrating a reduction in output voltage due to a load put on a circuit.}
\label{fig:pre_exp2}
\end{figure}

\newpage
\vspace{2mm}
\noindent
The current output from the ESR unit, see \figref{fig:pre_exp2} was be connected to a multimeter to measure the DC current. Then a passive circuit with a medium inductor coil was positioned close to a medium inductor coil attached to the basic ESR unit, see \figref{fig:pre_exp2}. By adjusting the frequency of the passive circuit, the current reading on the multimeter dropped. This therefore demonstrated that by removing energy from a resonant circuit placed a load on the circuit and the output voltage was reduced. After this principle was checked, the main experiment commenced.

\vspace{2mm}
\noindent
Upon successful completion of the preliminary checks, the DPPH sample was placed in one of the three coils provided and the experiment was set-up in the configuration as shown in \figref{fig:electron_picture}. 

\begin{figure}[h]
\centering
\includegraphics[scale=0.44]{"electron_picture".pdf}
\caption{The experimental set-up for investigating the electron spin resonance of DPPH.}
\label{fig:electron_picture}
\end{figure}

\vspace{2mm}
\noindent
Once all the relevant connections had been made to the oscilloscope and the ESR unit, the frequency was altered on the supplit unit for ESR over a range of frequencies from 15 to 130 MHz \cite{Paper02} at different resonances as shown in \figref{fig:resonance}. The range of frequencies were covered by using a selection of sizes for the coils to insert the DPPH sample: small, medium and large.

\begin{figure}[h]
\centering
\includegraphics[scale=0.4]{"resonance".pdf}
\caption{A picture showing the resonant frequency condition as shown on the oscilloscope trace.}
\label{fig:resonance}
\end{figure}


\vspace{2mm}
\noindent
Along with these resonant frequencies, the current and corresponding error in the current, obtained from the modulating current, were measured at different resonances from the radio frequency oscillator unit. These resonances were due to a radio frequency photon coming in and exiting an electron from one spin state to another. This takes energy out of the system and is observed as dips on the oscilloscope trace at resonance as shown in \figref{fig:resonance}. Once several resonance measurements for the current had been obtained, using the small, medium and large coils they were all repeated a total of three times and an average current, with corresponding error was taken. This was done to improve accuracy and reliability. Then \eqref{eq:magnetic-equation} was used to obtain the corresponding magnetic fields. These magnetic field measurements along with the corresponding frequency measurements were used in \eqref{eq:frequency-equation} to obtain a value for $g$ by plotting an appropriate straight-line graph. 
% Or you could use the individual results to obtain several values of $g$ and then averaging to obtain a single final value.


\subsection{Safety}
\label{ssec:safety}
During both experiments there were many sensitive electrical components used; such as a radio frequency oscillator, an oscolliscope and power supplies. Therefore, no food or drink were consumed whilst working in the laboratory and any exposed metal should not be touched when any power supply is switched on \cite{Paper01}. In addition to this,  the glass discharge tubes for the Atomic Spectroscopy experiment are fragile. So, when they required moving or changing the demonstrator or laboratory technician were present to handle these pieces of equipment. 


\section{Results}
\label{sec:results}

\subsection{Atomic Spectroscopy}
\label{ssec:atomic-results}

\figref{fig:Red_Plot} and \figref{fig:Blue_Plot} show a plot of $l$ against $\sqrt{d^2+l^2}$ for the two observed hydrogen transitions from the Balmer series: the Hydrogen $\alpha$, H$_\alpha$, and Hydrogen $\beta$, H$_\beta$, transitions.

\begin{figure}[h]
\centering
\includegraphics[scale=1.0]{"Red_Plot".pdf}
\caption{A graph of length $l$ against $\sqrt{d^2+l^2}$ for the H$_\alpha$ transition.}
\label{fig:Red_Plot}
\end{figure}

\begin{figure}[h]
\centering
\includegraphics[scale=1.0]{"Blue_Plot".pdf}
\caption{A graph of length $l$ against $\sqrt{d^2+l^2}$ for the H$_\beta$ transition.}
\label{fig:Blue_Plot}
\end{figure}

\newpage
\vspace{2mm}
\noindent
By determining the gradient from \figref{fig:Red_Plot} and \figref{fig:Blue_Plot} equation \eqref{eq:grating-equation} was used in the form 

\begin{equation}
\label{eq:wavelength}
\lambda = \sigma d_g
\end{equation}

\vspace{2mm}
\noindent
where $\sigma = \frac{l^2}{\sqrt{l^2+d^2}}$ is the value of the gradient. This equation was used to determine the wavelength for the H$_\alpha$ and H$_\beta$ transitions. The associated errors for the wavelengths were determined by using

\begin{equation}
\label{eq:wavelength-error}
\delta\lambda = \sqrt{\bigg(\frac{\partial{\lambda}}{\partial{\sigma}}\bigg)^2(\delta\sigma)^2 + \bigg(\frac{\partial{\lambda}}{\partial{d_g}}\bigg)^2(\delta d_g)^2}
\end{equation}

\vspace{2mm}
\noindent
where $\delta\lambda$ is the error in the wavelength (for H$_\alpha$ or H$_\beta$), $\delta\sigma$ is the error in the gradient and $\delta d_g$ is the error for the determined grating constant. The error for the grating constant was obtained by the usual propagation of errors routine  when determining the values of $l$ and $d$ to obtain $d_g$ as mentioned in \secref{ssec:atomic-method}. \tabref{tab:table1} shows the determined wavelengths from \figref{fig:Red_Plot} and \figref{fig:Blue_Plot} compared to the theoretical values obtained by performing calculations using equations \eqref{eq:balmer-series} and \eqref{eq:energy-wavelength} and using information from \cite{Paper01}.


\vspace{5mm}
\begin{table*}[h]
\centering % centre table
\caption{Table of Experimental Values and Theoretical Values for the Transition Wavelengths.}
\label{tab:table1}
\begin{tabular}{|c|c|c|}
\hline
Value & H$_\alpha$ Wavelength, $\lambda_{\alpha}$ (nm) & H$_\beta$ Wavelength, $\lambda_{\beta}$ (nm) \\
\hline
Experimental & $685 \pm 13$ & $507 \pm 8$ \\
\hline
Theoretical \cite{Paper01} & $658$ & $488$ \\
\hline
\end{tabular}
%\end{center}
\end{table*}


\vspace{2mm}
\noindent
Once the wavelengths had been determined, equation \eqref{eq:energy-wavelength} was used to find the energy emitted from these transitions. The associated uncertainty in the energy was then obtained by using the value $\delta\lambda$ in similar propagation methods used for determining the wavelength uncertainty. 

\vspace{2mm}
\noindent
Once having obtained the two energies associated with the two transitions, two values of the Rydberg constant could then be determined by using \eqref{eq:balmer-series} along with their uncertainties by using the uncertainty in the energy of the transitions. These two values were then averaged to determine the value more accurately

\begin{equation}
\label{eq:Rydberg-const}
Ry = \frac{Ry_{\alpha} + Ry_{\beta}}{2}
\end{equation}

\vspace{2mm}
\noindent
where $Ry_{\alpha}$ and $Ry_{\beta}$ are the two values of the Rydberg constant determined from the H$_\alpha$ and H$_\beta$ transitions respectively. The uncertainty was then determined by

\begin{equation}
\label{eq:Rydberg-error}
\delta Ry = \sqrt{\bigg(\frac{\partial{Ry}}{\partial{Ry_{\alpha}}}\bigg)^2 (\delta Ry_{\alpha})^2 + \bigg(\frac{\partial{Ry}}{\partial{Ry_{\beta}}}\bigg)^2 (\delta Ry_{\beta})^2}
\end{equation}

\vspace{2mm}
\noindent
where $\delta Ry_{\alpha}$ and $\delta Ry_{\beta}$ are the uncertainties for the two values of the Rydberg constant,  $Ry_{\alpha}$ and $Ry_{\beta}$ respectively. $\delta Ry$ is the final uncertainty for the final value of the Rydberg constant, $Ry$. \tabref{tab:table2} shows a caparison between the experimental and theoretical results of the Rydberg constant.

\vspace{5mm}
\begin{table*}[h]
\centering % centre table
\caption{Table of Calculated Values and Published Values for the Rydberg Constant}
\label{tab:table2}
\begin{tabular}{|c|c|c|}
\hline
Value & Rydberg Constant, $Ry$ (eV) \\
\hline
Experimental & $13.1 \pm 0.2$ \\
\hline
Theoretical \cite{Paper01} & $13.6$ \\
\hline
\end{tabular}
%\end{center}
\end{table*}

\subsection{Electron Spin Resonance}
\label{ssec:electron-results}

\figref{fig:Electron_Plot} shows the results from the determined magnetic fields at various currents from using \eqref{eq:magnetic-equation} plotted against different frequencies. This shows a direct linear relation between magnetic field and frequency; showing that an increase in frequency causes the magnetic field to increase as a result.

\begin{figure}[h]
\centering
\includegraphics[scale=1.0]{"Electron_plot1".pdf}
\caption{A graph of magnetic field, $B$ against frequency $\nu$.}
\label{fig:Electron_Plot}
\end{figure}

\newpage
\vspace{2mm}
\noindent
With reference to \figref{fig:Electron_Plot} equation \eqref{eq:frequency-equation} rearranges to the form of a straight-line graph

\begin{equation}
\label{eq:g-value}
B = \frac{h}{g\mu_B}\nu = \gamma\nu
\end{equation}

\vspace{2mm}
\noindent
where $\gamma = \frac{h}{g\mu_B}$ is the gradient of \figref{fig:Electron_Plot}. Thus, by obtaining the gradient of the graph and the gradient error, the value of $g$ can be determined. The associated uncertainty of $g$ can be obtained by using

\begin{equation}
\label{eq:g-error}
\delta g = \sqrt{\bigg(\frac{\partial{g}}{\partial{\gamma}}\bigg)^2 (\delta\gamma)^2}
\end{equation}

\vspace{2mm}
\noindent
where $\delta g$ is the uncertainty in $g$ and $\delta\gamma$ is the error in the gradient of \figref{fig:Electron_Plot}. \tabref{tab:table3} compares the values for $g$ determined experimentally to a theoretical value given by \cite{Paper02}.

\vspace{5mm}
\begin{table*}[h]
\centering % centre table
\caption{Table of Calculated Values and Published Values for the Spectroscopic Splitting Factor.}
\label{tab:table3}
\begin{tabular}{|c|c|c|}
\hline
Value & Spectroscopic Splitting Factor, $g$ \\
\hline
Experimental & $1.9733 \pm 0.0119$ \\
\hline
Theoretical \cite{Paper02} & $2.0023$ \\
\hline
\end{tabular}
%\end{center}
\end{table*}


\section{Discussion}
\label{sec:discussion}

Comparing the results in \secref{ssec:atomic-results} \tabref{tab:table2} it is clear to see that the experimental value for the Rydberg constant is close to the theoretical \cite{Paper01}. However, the theoretical value does not lie within the small uncertainty range of the experimental. Also \tabref{tab:table1} shows that the experimental values for the wavelengths are slightly different to those given in theory and again the theoretical values do not lie within the uncertainties of the experimental ones. So, these obtained wavelengths clearly impacted on the final result when obtaining the experimental value of the Rydberg constant and the small uncertainties found from the wavelengths then propagated through to give a small final uncertainty for the Rydberg constant.

\vspace{2mm}
\noindent
We can find the source of these small uncertainty values by inspecting \figref{fig:Red_Plot} and \figref{fig:Blue_Plot}. The error bars in these figures demonstrate that the uncertainties obtained from the experiment are small. When conducting the experiment, many of the uncertainties, such as the ones in the length $l$ and distance $d$ were had to estimate. But from analysis to \figref{fig:Red_Plot} and \figref{fig:Blue_Plot}, the vertical error bars are the smallest. This, therefore suggesting that more consideration should have been taken when estimating the uncertainty in the length, $l$, of the emission lines across the ruler as described in \secref{ssec:atomic-method}.

\vspace{2mm}
\noindent
In addition to this, when conducting the atomic spectroscopy experiment, many factors were hard to measure. For example, the emission lines could only be seen in a dark room. But, in this dark environment the length, $l$, across the ruler was hard to observe due to the dark back ground. Also, some of the emission lines were hard to view at certain distances. Some of the emission lines were faint at large distances, whilst at close distances the angle between $d$ and $l$ was large and this made viewing the emission lines difficult as described in \figref{fig:grating_plan}. Furthermore, the distance, $d$, of the diffraction grating from the ruler was also difficult to determine. This was because many factors, such as parallax, made it hard to obtain a result for this value; hence \figref{fig:Red_Plot} and \figref{fig:Blue_Plot} both show bigger horizontal error bars due to the bigger error in $d$ to accommodate for this discrepancy. All of these values would have impacted on the wavelength values in \tabref{tab:table1} due to using \eqref{eq:grating-equation} to obtain the wavelengths. Hence, the obtained wavelength and associated error would have led to the discrepancies and small error in the experimental value of the Rydberg constant as shown tab{tab:table1}. 

\vspace{2mm}
\noindent
By considering \tabref{tab:table3} we can see that the value of the spectroscopic splitting factor, $g$, obtained experimentally different to the one given in theory \cite{Paper02}. Also, the theoretical value does not lie within the uncertainty range of the the experimental therefore showing some discrepancies within the experiment. By looking at \figref{fig:Electron_Plot} we can see that the linear fitting line, obtained by the least squares routine, is very accurate and passes through most of the points. This therefore led to a small error in the gradient of the line, $\delta\gamma$ and when propagated through using \eqref{eq:g-error} the gave a small error in $g$. This again, therefore implies that errors within the experiment were small. A possible reason for these small errors is due to the experiment being repeated many times. This reduced the overall error (within the current) and improved confidence when obtaining the values of the current, and hence the magnetic field. Therefore, the error in the result on $g$ is reduced. 

\vspace{2mm}
\noindent
By further inspection, the actual experimental value of $g$ is slightly different to what was predicted in theory. When conducting the experiment, numerous factors were hard to determine. For example, the resonance condition was hard to measure. It was hard to get the period of resonance exactly right due to the difficulty in stabilising the frequency from the supply unit. Also, when the resonance conditions were displayed on the oscilloscope, it was hard to deduce exact values of current as there were discrepancies in viewing the period on the grid lines of the screen as shown in \figref{fig:resonance}. By repeating the experiment three times, this reduced some of the inaccuracies within the measurements, but also showed that there was variation within the measured current. This then gave rise to some variation in the magnetic field as shown in the vertical error bars in \figref{fig:Electron_Plot}. 

\vspace{2mm}
\noindent
In addition to viewing the resonance conditions, when setting up the circuit, the distance between the Helmholtz coils was slightly different to that of the radius of a coil, as mentioned in \secref{ssec:electron-method}. This was because the radio frequency oscillator had to fit between the two coils in order to ensure that the DPPH was central in position as shown in \figref{fig:electron_picture}. Therefore, the actual measured value of $r$ for use in \eqref{eq:magnetic-equation} was now $7.5 \pm0.7$ cm. The large error was used to ensure that had the value of $6.75$cm in the range and to accommodate parallax error when setting up the coils. This was all noted and used in equation  \eqref{eq:magnetic-equation} when determining generated the magnetic fields. Therefore, this would have impacted on the values of $g$. If more time had been permitted, in order to have obtained a more accurate value of $g$ a different relation would have needed to have been derived for the magnetic field using the Biot-Savhart law. This would have taken into account the different distance between the two Helmholtz coils and hence would have generated a more reliable value of $g$.


\section{Conclusion}
\label{sec:conclusion}
Overall, both experiments gave some fairly good values for the Rydberg constant, $Ry$ and spectroscopic splitting factor $g$. Both of the experiments demonstrate some interesting properties associated with electron energy changes. The atomic spectroscopy experiment demonstrated how electrons change energy levels within the hydrogen atom and the electron spin resonance experiment showed how electrons change spin when exposed to a magnetic field. By taking into account careful estimation of errors and experimental set-ups the experimental results obtained in \secref{ssec:atomic-results} \tabref{tab:table2} and \secref{ssec:electron-results} \tabref{tab:table3} would have been more reliable. This would have then showed a stronger correlation between the experimental values and those predicted in theory. However, it is clear to see that these values obtained  for $Ry$ and $g$ from the experiments show strong correlations to those given in theory. This verifies that these physical constants are indeed valid when considering atomic and subatomic physics. 


\begin{thebibliography}{9}
\bibitem{Paper01} Atomic Spectroscopy worksheet.
\bibitem{Paper02} Electron Spin Resonance worksheet. 
\bibitem{Book01} Young and Freedman, \textit{University Physics}, 13\textsuperscript{th} Edition, Chapter 41, pages 1519 - 1527.
\bibitem{Book02} Young and Freedman, \textit{University Physics}, 13\textsuperscript{th} Edition, Chapter 39, pages 1427 - 1437.
%\bibitem{Book02} Young and Freedman, \textit{University Physics}, 13\textsuperscript{th} Edition, pages 1211--1212.
%\bibitem{Web01} \url{http://simbad.u-strasbg.fr/simbad/}
%\bibitem{Web02} \url{http://ecommons.luc.edu/cgi/viewcontent.cgi?article=1667&context=luc_theses} See page 22 for refractive index values.
\end{thebibliography}

\end{document}